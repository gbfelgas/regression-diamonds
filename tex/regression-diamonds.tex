
% Default to the notebook output style

    


% Inherit from the specified cell style.




    
\documentclass[11pt]{article}

    
    
    \usepackage[T1]{fontenc}
    % Nicer default font (+ math font) than Computer Modern for most use cases
    \usepackage{mathpazo}

    % Basic figure setup, for now with no caption control since it's done
    % automatically by Pandoc (which extracts ![](path) syntax from Markdown).
    \usepackage{graphicx}
    % We will generate all images so they have a width \maxwidth. This means
    % that they will get their normal width if they fit onto the page, but
    % are scaled down if they would overflow the margins.
    \makeatletter
    \def\maxwidth{\ifdim\Gin@nat@width>\linewidth\linewidth
    \else\Gin@nat@width\fi}
    \makeatother
    \let\Oldincludegraphics\includegraphics
    % Set max figure width to be 80% of text width, for now hardcoded.
    \renewcommand{\includegraphics}[1]{\Oldincludegraphics[width=.8\maxwidth]{#1}}
    % Ensure that by default, figures have no caption (until we provide a
    % proper Figure object with a Caption API and a way to capture that
    % in the conversion process - todo).
    \usepackage{caption}
    \DeclareCaptionLabelFormat{nolabel}{}
    \captionsetup{labelformat=nolabel}

    \usepackage{adjustbox} % Used to constrain images to a maximum size 
    \usepackage{xcolor} % Allow colors to be defined
    \usepackage{enumerate} % Needed for markdown enumerations to work
    \usepackage{geometry} % Used to adjust the document margins
    \usepackage{amsmath} % Equations
    \usepackage{amssymb} % Equations
    \usepackage{textcomp} % defines textquotesingle
    % Hack from http://tex.stackexchange.com/a/47451/13684:
    \AtBeginDocument{%
        \def\PYZsq{\textquotesingle}% Upright quotes in Pygmentized code
    }
    \usepackage{upquote} % Upright quotes for verbatim code
    \usepackage{eurosym} % defines \euro
    \usepackage[mathletters]{ucs} % Extended unicode (utf-8) support
    \usepackage[utf8x]{inputenc} % Allow utf-8 characters in the tex document
    \usepackage{fancyvrb} % verbatim replacement that allows latex
    \usepackage{grffile} % extends the file name processing of package graphics 
                         % to support a larger range 
    % The hyperref package gives us a pdf with properly built
    % internal navigation ('pdf bookmarks' for the table of contents,
    % internal cross-reference links, web links for URLs, etc.)
    \usepackage{hyperref}
    \usepackage{longtable} % longtable support required by pandoc >1.10
    \usepackage{booktabs}  % table support for pandoc > 1.12.2
    \usepackage[inline]{enumitem} % IRkernel/repr support (it uses the enumerate* environment)
    \usepackage[normalem]{ulem} % ulem is needed to support strikethroughs (\sout)
                                % normalem makes italics be italics, not underlines
    \usepackage{mathrsfs}
    

    
    
    % Colors for the hyperref package
    \definecolor{urlcolor}{rgb}{0,.145,.698}
    \definecolor{linkcolor}{rgb}{.71,0.21,0.01}
    \definecolor{citecolor}{rgb}{.12,.54,.11}

    % ANSI colors
    \definecolor{ansi-black}{HTML}{3E424D}
    \definecolor{ansi-black-intense}{HTML}{282C36}
    \definecolor{ansi-red}{HTML}{E75C58}
    \definecolor{ansi-red-intense}{HTML}{B22B31}
    \definecolor{ansi-green}{HTML}{00A250}
    \definecolor{ansi-green-intense}{HTML}{007427}
    \definecolor{ansi-yellow}{HTML}{DDB62B}
    \definecolor{ansi-yellow-intense}{HTML}{B27D12}
    \definecolor{ansi-blue}{HTML}{208FFB}
    \definecolor{ansi-blue-intense}{HTML}{0065CA}
    \definecolor{ansi-magenta}{HTML}{D160C4}
    \definecolor{ansi-magenta-intense}{HTML}{A03196}
    \definecolor{ansi-cyan}{HTML}{60C6C8}
    \definecolor{ansi-cyan-intense}{HTML}{258F8F}
    \definecolor{ansi-white}{HTML}{C5C1B4}
    \definecolor{ansi-white-intense}{HTML}{A1A6B2}
    \definecolor{ansi-default-inverse-fg}{HTML}{FFFFFF}
    \definecolor{ansi-default-inverse-bg}{HTML}{000000}

    % commands and environments needed by pandoc snippets
    % extracted from the output of `pandoc -s`
    \providecommand{\tightlist}{%
      \setlength{\itemsep}{0pt}\setlength{\parskip}{0pt}}
    \DefineVerbatimEnvironment{Highlighting}{Verbatim}{commandchars=\\\{\}}
    % Add ',fontsize=\small' for more characters per line
    \newenvironment{Shaded}{}{}
    \newcommand{\KeywordTok}[1]{\textcolor[rgb]{0.00,0.44,0.13}{\textbf{{#1}}}}
    \newcommand{\DataTypeTok}[1]{\textcolor[rgb]{0.56,0.13,0.00}{{#1}}}
    \newcommand{\DecValTok}[1]{\textcolor[rgb]{0.25,0.63,0.44}{{#1}}}
    \newcommand{\BaseNTok}[1]{\textcolor[rgb]{0.25,0.63,0.44}{{#1}}}
    \newcommand{\FloatTok}[1]{\textcolor[rgb]{0.25,0.63,0.44}{{#1}}}
    \newcommand{\CharTok}[1]{\textcolor[rgb]{0.25,0.44,0.63}{{#1}}}
    \newcommand{\StringTok}[1]{\textcolor[rgb]{0.25,0.44,0.63}{{#1}}}
    \newcommand{\CommentTok}[1]{\textcolor[rgb]{0.38,0.63,0.69}{\textit{{#1}}}}
    \newcommand{\OtherTok}[1]{\textcolor[rgb]{0.00,0.44,0.13}{{#1}}}
    \newcommand{\AlertTok}[1]{\textcolor[rgb]{1.00,0.00,0.00}{\textbf{{#1}}}}
    \newcommand{\FunctionTok}[1]{\textcolor[rgb]{0.02,0.16,0.49}{{#1}}}
    \newcommand{\RegionMarkerTok}[1]{{#1}}
    \newcommand{\ErrorTok}[1]{\textcolor[rgb]{1.00,0.00,0.00}{\textbf{{#1}}}}
    \newcommand{\NormalTok}[1]{{#1}}
    
    % Additional commands for more recent versions of Pandoc
    \newcommand{\ConstantTok}[1]{\textcolor[rgb]{0.53,0.00,0.00}{{#1}}}
    \newcommand{\SpecialCharTok}[1]{\textcolor[rgb]{0.25,0.44,0.63}{{#1}}}
    \newcommand{\VerbatimStringTok}[1]{\textcolor[rgb]{0.25,0.44,0.63}{{#1}}}
    \newcommand{\SpecialStringTok}[1]{\textcolor[rgb]{0.73,0.40,0.53}{{#1}}}
    \newcommand{\ImportTok}[1]{{#1}}
    \newcommand{\DocumentationTok}[1]{\textcolor[rgb]{0.73,0.13,0.13}{\textit{{#1}}}}
    \newcommand{\AnnotationTok}[1]{\textcolor[rgb]{0.38,0.63,0.69}{\textbf{\textit{{#1}}}}}
    \newcommand{\CommentVarTok}[1]{\textcolor[rgb]{0.38,0.63,0.69}{\textbf{\textit{{#1}}}}}
    \newcommand{\VariableTok}[1]{\textcolor[rgb]{0.10,0.09,0.49}{{#1}}}
    \newcommand{\ControlFlowTok}[1]{\textcolor[rgb]{0.00,0.44,0.13}{\textbf{{#1}}}}
    \newcommand{\OperatorTok}[1]{\textcolor[rgb]{0.40,0.40,0.40}{{#1}}}
    \newcommand{\BuiltInTok}[1]{{#1}}
    \newcommand{\ExtensionTok}[1]{{#1}}
    \newcommand{\PreprocessorTok}[1]{\textcolor[rgb]{0.74,0.48,0.00}{{#1}}}
    \newcommand{\AttributeTok}[1]{\textcolor[rgb]{0.49,0.56,0.16}{{#1}}}
    \newcommand{\InformationTok}[1]{\textcolor[rgb]{0.38,0.63,0.69}{\textbf{\textit{{#1}}}}}
    \newcommand{\WarningTok}[1]{\textcolor[rgb]{0.38,0.63,0.69}{\textbf{\textit{{#1}}}}}
    
    
    % Define a nice break command that doesn't care if a line doesn't already
    % exist.
    \def\br{\hspace*{\fill} \\* }
    % Math Jax compatibility definitions
    \def\gt{>}
    \def\lt{<}
    \let\Oldtex\TeX
    \let\Oldlatex\LaTeX
    \renewcommand{\TeX}{\textrm{\Oldtex}}
    \renewcommand{\LaTeX}{\textrm{\Oldlatex}}
    % Document parameters
    % Document title
    \title{regression-diamonds}
    
    
    
    
    

    % Pygments definitions
    
\makeatletter
\def\PY@reset{\let\PY@it=\relax \let\PY@bf=\relax%
    \let\PY@ul=\relax \let\PY@tc=\relax%
    \let\PY@bc=\relax \let\PY@ff=\relax}
\def\PY@tok#1{\csname PY@tok@#1\endcsname}
\def\PY@toks#1+{\ifx\relax#1\empty\else%
    \PY@tok{#1}\expandafter\PY@toks\fi}
\def\PY@do#1{\PY@bc{\PY@tc{\PY@ul{%
    \PY@it{\PY@bf{\PY@ff{#1}}}}}}}
\def\PY#1#2{\PY@reset\PY@toks#1+\relax+\PY@do{#2}}

\expandafter\def\csname PY@tok@w\endcsname{\def\PY@tc##1{\textcolor[rgb]{0.73,0.73,0.73}{##1}}}
\expandafter\def\csname PY@tok@c\endcsname{\let\PY@it=\textit\def\PY@tc##1{\textcolor[rgb]{0.25,0.50,0.50}{##1}}}
\expandafter\def\csname PY@tok@cp\endcsname{\def\PY@tc##1{\textcolor[rgb]{0.74,0.48,0.00}{##1}}}
\expandafter\def\csname PY@tok@k\endcsname{\let\PY@bf=\textbf\def\PY@tc##1{\textcolor[rgb]{0.00,0.50,0.00}{##1}}}
\expandafter\def\csname PY@tok@kp\endcsname{\def\PY@tc##1{\textcolor[rgb]{0.00,0.50,0.00}{##1}}}
\expandafter\def\csname PY@tok@kt\endcsname{\def\PY@tc##1{\textcolor[rgb]{0.69,0.00,0.25}{##1}}}
\expandafter\def\csname PY@tok@o\endcsname{\def\PY@tc##1{\textcolor[rgb]{0.40,0.40,0.40}{##1}}}
\expandafter\def\csname PY@tok@ow\endcsname{\let\PY@bf=\textbf\def\PY@tc##1{\textcolor[rgb]{0.67,0.13,1.00}{##1}}}
\expandafter\def\csname PY@tok@nb\endcsname{\def\PY@tc##1{\textcolor[rgb]{0.00,0.50,0.00}{##1}}}
\expandafter\def\csname PY@tok@nf\endcsname{\def\PY@tc##1{\textcolor[rgb]{0.00,0.00,1.00}{##1}}}
\expandafter\def\csname PY@tok@nc\endcsname{\let\PY@bf=\textbf\def\PY@tc##1{\textcolor[rgb]{0.00,0.00,1.00}{##1}}}
\expandafter\def\csname PY@tok@nn\endcsname{\let\PY@bf=\textbf\def\PY@tc##1{\textcolor[rgb]{0.00,0.00,1.00}{##1}}}
\expandafter\def\csname PY@tok@ne\endcsname{\let\PY@bf=\textbf\def\PY@tc##1{\textcolor[rgb]{0.82,0.25,0.23}{##1}}}
\expandafter\def\csname PY@tok@nv\endcsname{\def\PY@tc##1{\textcolor[rgb]{0.10,0.09,0.49}{##1}}}
\expandafter\def\csname PY@tok@no\endcsname{\def\PY@tc##1{\textcolor[rgb]{0.53,0.00,0.00}{##1}}}
\expandafter\def\csname PY@tok@nl\endcsname{\def\PY@tc##1{\textcolor[rgb]{0.63,0.63,0.00}{##1}}}
\expandafter\def\csname PY@tok@ni\endcsname{\let\PY@bf=\textbf\def\PY@tc##1{\textcolor[rgb]{0.60,0.60,0.60}{##1}}}
\expandafter\def\csname PY@tok@na\endcsname{\def\PY@tc##1{\textcolor[rgb]{0.49,0.56,0.16}{##1}}}
\expandafter\def\csname PY@tok@nt\endcsname{\let\PY@bf=\textbf\def\PY@tc##1{\textcolor[rgb]{0.00,0.50,0.00}{##1}}}
\expandafter\def\csname PY@tok@nd\endcsname{\def\PY@tc##1{\textcolor[rgb]{0.67,0.13,1.00}{##1}}}
\expandafter\def\csname PY@tok@s\endcsname{\def\PY@tc##1{\textcolor[rgb]{0.73,0.13,0.13}{##1}}}
\expandafter\def\csname PY@tok@sd\endcsname{\let\PY@it=\textit\def\PY@tc##1{\textcolor[rgb]{0.73,0.13,0.13}{##1}}}
\expandafter\def\csname PY@tok@si\endcsname{\let\PY@bf=\textbf\def\PY@tc##1{\textcolor[rgb]{0.73,0.40,0.53}{##1}}}
\expandafter\def\csname PY@tok@se\endcsname{\let\PY@bf=\textbf\def\PY@tc##1{\textcolor[rgb]{0.73,0.40,0.13}{##1}}}
\expandafter\def\csname PY@tok@sr\endcsname{\def\PY@tc##1{\textcolor[rgb]{0.73,0.40,0.53}{##1}}}
\expandafter\def\csname PY@tok@ss\endcsname{\def\PY@tc##1{\textcolor[rgb]{0.10,0.09,0.49}{##1}}}
\expandafter\def\csname PY@tok@sx\endcsname{\def\PY@tc##1{\textcolor[rgb]{0.00,0.50,0.00}{##1}}}
\expandafter\def\csname PY@tok@m\endcsname{\def\PY@tc##1{\textcolor[rgb]{0.40,0.40,0.40}{##1}}}
\expandafter\def\csname PY@tok@gh\endcsname{\let\PY@bf=\textbf\def\PY@tc##1{\textcolor[rgb]{0.00,0.00,0.50}{##1}}}
\expandafter\def\csname PY@tok@gu\endcsname{\let\PY@bf=\textbf\def\PY@tc##1{\textcolor[rgb]{0.50,0.00,0.50}{##1}}}
\expandafter\def\csname PY@tok@gd\endcsname{\def\PY@tc##1{\textcolor[rgb]{0.63,0.00,0.00}{##1}}}
\expandafter\def\csname PY@tok@gi\endcsname{\def\PY@tc##1{\textcolor[rgb]{0.00,0.63,0.00}{##1}}}
\expandafter\def\csname PY@tok@gr\endcsname{\def\PY@tc##1{\textcolor[rgb]{1.00,0.00,0.00}{##1}}}
\expandafter\def\csname PY@tok@ge\endcsname{\let\PY@it=\textit}
\expandafter\def\csname PY@tok@gs\endcsname{\let\PY@bf=\textbf}
\expandafter\def\csname PY@tok@gp\endcsname{\let\PY@bf=\textbf\def\PY@tc##1{\textcolor[rgb]{0.00,0.00,0.50}{##1}}}
\expandafter\def\csname PY@tok@go\endcsname{\def\PY@tc##1{\textcolor[rgb]{0.53,0.53,0.53}{##1}}}
\expandafter\def\csname PY@tok@gt\endcsname{\def\PY@tc##1{\textcolor[rgb]{0.00,0.27,0.87}{##1}}}
\expandafter\def\csname PY@tok@err\endcsname{\def\PY@bc##1{\setlength{\fboxsep}{0pt}\fcolorbox[rgb]{1.00,0.00,0.00}{1,1,1}{\strut ##1}}}
\expandafter\def\csname PY@tok@kc\endcsname{\let\PY@bf=\textbf\def\PY@tc##1{\textcolor[rgb]{0.00,0.50,0.00}{##1}}}
\expandafter\def\csname PY@tok@kd\endcsname{\let\PY@bf=\textbf\def\PY@tc##1{\textcolor[rgb]{0.00,0.50,0.00}{##1}}}
\expandafter\def\csname PY@tok@kn\endcsname{\let\PY@bf=\textbf\def\PY@tc##1{\textcolor[rgb]{0.00,0.50,0.00}{##1}}}
\expandafter\def\csname PY@tok@kr\endcsname{\let\PY@bf=\textbf\def\PY@tc##1{\textcolor[rgb]{0.00,0.50,0.00}{##1}}}
\expandafter\def\csname PY@tok@bp\endcsname{\def\PY@tc##1{\textcolor[rgb]{0.00,0.50,0.00}{##1}}}
\expandafter\def\csname PY@tok@fm\endcsname{\def\PY@tc##1{\textcolor[rgb]{0.00,0.00,1.00}{##1}}}
\expandafter\def\csname PY@tok@vc\endcsname{\def\PY@tc##1{\textcolor[rgb]{0.10,0.09,0.49}{##1}}}
\expandafter\def\csname PY@tok@vg\endcsname{\def\PY@tc##1{\textcolor[rgb]{0.10,0.09,0.49}{##1}}}
\expandafter\def\csname PY@tok@vi\endcsname{\def\PY@tc##1{\textcolor[rgb]{0.10,0.09,0.49}{##1}}}
\expandafter\def\csname PY@tok@vm\endcsname{\def\PY@tc##1{\textcolor[rgb]{0.10,0.09,0.49}{##1}}}
\expandafter\def\csname PY@tok@sa\endcsname{\def\PY@tc##1{\textcolor[rgb]{0.73,0.13,0.13}{##1}}}
\expandafter\def\csname PY@tok@sb\endcsname{\def\PY@tc##1{\textcolor[rgb]{0.73,0.13,0.13}{##1}}}
\expandafter\def\csname PY@tok@sc\endcsname{\def\PY@tc##1{\textcolor[rgb]{0.73,0.13,0.13}{##1}}}
\expandafter\def\csname PY@tok@dl\endcsname{\def\PY@tc##1{\textcolor[rgb]{0.73,0.13,0.13}{##1}}}
\expandafter\def\csname PY@tok@s2\endcsname{\def\PY@tc##1{\textcolor[rgb]{0.73,0.13,0.13}{##1}}}
\expandafter\def\csname PY@tok@sh\endcsname{\def\PY@tc##1{\textcolor[rgb]{0.73,0.13,0.13}{##1}}}
\expandafter\def\csname PY@tok@s1\endcsname{\def\PY@tc##1{\textcolor[rgb]{0.73,0.13,0.13}{##1}}}
\expandafter\def\csname PY@tok@mb\endcsname{\def\PY@tc##1{\textcolor[rgb]{0.40,0.40,0.40}{##1}}}
\expandafter\def\csname PY@tok@mf\endcsname{\def\PY@tc##1{\textcolor[rgb]{0.40,0.40,0.40}{##1}}}
\expandafter\def\csname PY@tok@mh\endcsname{\def\PY@tc##1{\textcolor[rgb]{0.40,0.40,0.40}{##1}}}
\expandafter\def\csname PY@tok@mi\endcsname{\def\PY@tc##1{\textcolor[rgb]{0.40,0.40,0.40}{##1}}}
\expandafter\def\csname PY@tok@il\endcsname{\def\PY@tc##1{\textcolor[rgb]{0.40,0.40,0.40}{##1}}}
\expandafter\def\csname PY@tok@mo\endcsname{\def\PY@tc##1{\textcolor[rgb]{0.40,0.40,0.40}{##1}}}
\expandafter\def\csname PY@tok@ch\endcsname{\let\PY@it=\textit\def\PY@tc##1{\textcolor[rgb]{0.25,0.50,0.50}{##1}}}
\expandafter\def\csname PY@tok@cm\endcsname{\let\PY@it=\textit\def\PY@tc##1{\textcolor[rgb]{0.25,0.50,0.50}{##1}}}
\expandafter\def\csname PY@tok@cpf\endcsname{\let\PY@it=\textit\def\PY@tc##1{\textcolor[rgb]{0.25,0.50,0.50}{##1}}}
\expandafter\def\csname PY@tok@c1\endcsname{\let\PY@it=\textit\def\PY@tc##1{\textcolor[rgb]{0.25,0.50,0.50}{##1}}}
\expandafter\def\csname PY@tok@cs\endcsname{\let\PY@it=\textit\def\PY@tc##1{\textcolor[rgb]{0.25,0.50,0.50}{##1}}}

\def\PYZbs{\char`\\}
\def\PYZus{\char`\_}
\def\PYZob{\char`\{}
\def\PYZcb{\char`\}}
\def\PYZca{\char`\^}
\def\PYZam{\char`\&}
\def\PYZlt{\char`\<}
\def\PYZgt{\char`\>}
\def\PYZsh{\char`\#}
\def\PYZpc{\char`\%}
\def\PYZdl{\char`\$}
\def\PYZhy{\char`\-}
\def\PYZsq{\char`\'}
\def\PYZdq{\char`\"}
\def\PYZti{\char`\~}
% for compatibility with earlier versions
\def\PYZat{@}
\def\PYZlb{[}
\def\PYZrb{]}
\makeatother


    % Exact colors from NB
    \definecolor{incolor}{rgb}{0.0, 0.0, 0.5}
    \definecolor{outcolor}{rgb}{0.545, 0.0, 0.0}



    
    % Prevent overflowing lines due to hard-to-break entities
    \sloppy 
    % Setup hyperref package
    \hypersetup{
      breaklinks=true,  % so long urls are correctly broken across lines
      colorlinks=true,
      urlcolor=urlcolor,
      linkcolor=linkcolor,
      citecolor=citecolor,
      }
    % Slightly bigger margins than the latex defaults
    
    \geometry{verbose,tmargin=1in,bmargin=1in,lmargin=1in,rmargin=1in}
    
    

    \begin{document}
    
    
    \maketitle
    
    

    
    \hypertarget{trabalho-1---regressuxe3o-multivariuxe1vel}{%
\section{Trabalho 1 - Regressão
Multivariável}\label{trabalho-1---regressuxe3o-multivariuxe1vel}}

\hypertarget{estimativa-de-preuxe7os-de-diamantes-de-acordo-com-suas-caracteruxedsticas.}{%
\section{Estimativa de preços de diamantes de acordo com suas
características.}\label{estimativa-de-preuxe7os-de-diamantes-de-acordo-com-suas-caracteruxedsticas.}}

UFRJ/POLI/DEL - Introdução ao Aprendizado de Máquina (EEL891)\\
Prof.~Heraldo Almeira - Julho de 2019 Maria Gabriella Andrade Felgas
(DRE: 111471809)

    \hypertarget{introduuxe7uxe3o}{%
\section{Introdução}\label{introduuxe7uxe3o}}

    Este trabalho tem como objetivo desenvolver um modelo de regressão para
estimar os preços de diamantes a partir de características específicas
dadas. Para realizá-lo, foram disponibilizados um conjunto de dados de
treino, com alvo, e um conjunto de dados de teste, cujo alvo deve ser
estimado pela aluna, além de um modelo do arquivo a ser submetido à
competição.

Cada atributo do conjunto de dados está descrito abaixo:

\begin{itemize}
\item
  \textbf{id}: Identificação única do diamante;
\item
  \textbf{carat}: Peso em quilates (1 quilate = 0,2 g);
\item
  \textbf{cut}: Qualidade da lapidação, em uma escala categórica ordinal
  com os seguinte valores: \textgreater{}- \textbf{``Fair''} = Aceitável
  (classificação de menor valor); \textgreater{}- \textbf{``Good''} =
  Boa; \textgreater{}- \textbf{``Very Good''} = Muito boa;
  \textgreater{}- \textbf{``Premium''} = Excelente; \textgreater{}-
  \textbf{``Ideal''} = Perfeita (classificação de maior valor).
\item
  \textbf{color}: Cor, em uma escala categórica ordinal com os seguintes
  valores: \textgreater{}- \textbf{``D''} = Excepcionalmente incolor
  extra (classificação de maior valor); \textgreater{}- \textbf{``E''} =
  Excepcionalmente incolor; \textgreater{}- \textbf{``F''} =
  Perfeitamente incolor; \textgreater{}- \textbf{``G''} = Nitidamente
  incolor; \textgreater{}- \textbf{``H''} = Incolor; \textgreater{}-
  \textbf{``I''} = Cor levemente perceptível; \textgreater{}-
  \textbf{``J''} = Cor perceptível (classificação de menor valor).
\item
  \textbf{clarity}: Pureza, em uma escala categórica ordinal com os
  seguintes valores: \textgreater{}- \textbf{``I1''} = Inclusões
  evidentes com lupa de 10x (classificação de menor valor);
  \textgreater{}- \textbf{``SI2''} e \textbf{``SI1''} = Inclusões
  pequenas, mas fáceis de serem visualizadas com lupa de 10x;
  \textgreater{}- \textbf{``VS2''} e \textbf{``VS1''} = Inclusões muito
  pequenas e difíceis de serem visualizadas com lupa de 10x;
  \textgreater{}- \textbf{``VVS2''} e \textbf{``VVS1''} = Inclusões
  extremamente pequenas e muito difíceis de serem visualizadas com lupa
  de 10x; \textgreater{}- \textbf{``IF''} = Livre de inclusões
  (classificação de maior valor).
\item
  \textbf{x}: Comprimento em milímetros;
\item
  \textbf{y}: Largura em milímetros;
\item
  \textbf{z}: Profundidade em milímetros;
\item
  \textbf{depth}: Profundidade relativa = 100 * z / mean(x,y) = 200 * z
  / ( x + y );
\item
  \textbf{table}: Razão percentual entre entre a largura no topo e a
  largura no ponto mais largo;
\item ~
  \hypertarget{price-preuxe7o-do-diamante-em-duxf3lares-americanos.}{%
  \subsection{\texorpdfstring{\textbf{price}: Preço do diamante, em
  dólares
  americanos.}{price: Preço do diamante, em dólares americanos.}}\label{price-preuxe7o-do-diamante-em-duxf3lares-americanos.}}

  \textbf{OBS:} Este documento apresenta partes com código comentado
  devido aos testes realizados durante o desenvolvimento do modelo.
  Foram mantidos para melhor compreensão da lógica utilizada e para
  reprodução, a quem interessar.
\end{itemize}

    \hypertarget{importando-as-bibliotecas-e-ferramentas}{%
\section{Importando as Bibliotecas e
Ferramentas}\label{importando-as-bibliotecas-e-ferramentas}}

    Para realizar este trabalho, foi necessário utilizar diversas
bibliotecas disponíveis em Python:

\begin{itemize}
\tightlist
\item
  \textbf{Processamento e manipulação de dados:} Numpy e Pandas;
\item
  \textbf{Visualização de dados:} Matplotlib e Seaborn;
\item
  \textbf{Modelos de treinamento, ferramentas e métricas:} Scikit-learn.
\end{itemize}

    \begin{Verbatim}[commandchars=\\\{\}]
{\color{incolor}In [{\color{incolor}188}]:} \PY{c+c1}{\PYZsh{} Importando as bibliotecas e setando o ambiente de desenvolvimento}
          
          \PY{c+c1}{\PYZsh{} Bibliotecas para processamento e manipulacao dos dados}
          \PY{k+kn}{import} \PY{n+nn}{numpy} \PY{k}{as} \PY{n+nn}{np}
          \PY{k+kn}{import} \PY{n+nn}{pandas} \PY{k}{as} \PY{n+nn}{pd}
          
          \PY{c+c1}{\PYZsh{} Bibliotecas para visualizacao dos dados}
          \PY{k+kn}{import} \PY{n+nn}{matplotlib}\PY{n+nn}{.}\PY{n+nn}{pyplot} \PY{k}{as} \PY{n+nn}{plt}
          \PY{k+kn}{import} \PY{n+nn}{seaborn} \PY{k}{as} \PY{n+nn}{sns}
          
          \PY{c+c1}{\PYZsh{} Bibliotecas dos modelos de treinamento}
          \PY{k+kn}{from} \PY{n+nn}{sklearn}\PY{n+nn}{.}\PY{n+nn}{linear\PYZus{}model} \PY{k}{import} \PY{n}{LinearRegression}\PY{p}{,} \PY{n}{Ridge}\PY{p}{,} \PY{n}{Lasso}
          \PY{k+kn}{from} \PY{n+nn}{sklearn}\PY{n+nn}{.}\PY{n+nn}{ensemble} \PY{k}{import} \PY{n}{RandomForestRegressor}\PY{p}{,} \PY{n}{GradientBoostingRegressor}\PY{p}{,} \PY{n}{AdaBoostRegressor}\PY{p}{,} \PY{n}{ExtraTreesRegressor}
          
          \PY{c+c1}{\PYZsh{} Bibliotecas de ferramentas e métricas}
          \PY{k+kn}{from} \PY{n+nn}{sklearn}\PY{n+nn}{.}\PY{n+nn}{model\PYZus{}selection} \PY{k}{import} \PY{n}{train\PYZus{}test\PYZus{}split}\PY{p}{,} \PY{n}{RandomizedSearchCV}\PY{p}{,} \PY{n}{GridSearchCV}\PY{p}{,} \PY{n}{cross\PYZus{}val\PYZus{}score}\PY{p}{,} \PY{n}{cross\PYZus{}validate}
          \PY{k+kn}{from} \PY{n+nn}{sklearn}\PY{n+nn}{.}\PY{n+nn}{metrics} \PY{k}{import} \PY{n}{mean\PYZus{}squared\PYZus{}error}\PY{p}{,} \PY{n}{r2\PYZus{}score}\PY{p}{,} \PY{n}{mean\PYZus{}absolute\PYZus{}error}\PY{p}{,} \PY{n}{make\PYZus{}scorer}
\end{Verbatim}

    \hypertarget{minerauxe7uxe3o-e-anuxe1lise-de-dados}{%
\section{Mineração e Análise de
Dados}\label{minerauxe7uxe3o-e-anuxe1lise-de-dados}}

    A seguir, segue o passo a passo para analisar e tratar o conjunto de
dados de acordo com as observações.

    \hypertarget{carregando-conjunto-de-treino}{%
\subsubsection{Carregando Conjunto de
Treino}\label{carregando-conjunto-de-treino}}

    \begin{Verbatim}[commandchars=\\\{\}]
{\color{incolor}In [{\color{incolor}189}]:} \PY{c+c1}{\PYZsh{} Carregando os dados de treino como dataframe}
          \PY{c+c1}{\PYZsh{} e observando os atributos}
          \PY{n}{train} \PY{o}{=} \PY{n}{pd}\PY{o}{.}\PY{n}{read\PYZus{}csv}\PY{p}{(}\PY{l+s+s1}{\PYZsq{}}\PY{l+s+s1}{data/train.csv}\PY{l+s+s1}{\PYZsq{}}\PY{p}{)}
          \PY{n}{train}\PY{o}{.}\PY{n}{head}\PY{p}{(}\PY{p}{)}
\end{Verbatim}

\begin{Verbatim}[commandchars=\\\{\}]
{\color{outcolor}Out[{\color{outcolor}189}]:}       id  carat        cut color clarity     x     y     z  depth  table  \textbackslash{}
          0  20000   0.35  Very Good     G    VVS2  4.44  4.48  2.80   62.8   58.0   
          1  20001   0.70      Ideal     F     SI1  5.66  5.69  3.55   62.6   56.0   
          2  20002   0.32      Ideal     F    VVS1  4.42  4.38  2.70   61.4   56.0   
          3  20003   0.30      Ideal     H    VVS2  4.32  4.35  2.67   61.7   54.2   
          4  20004   0.33    Premium     I    VVS2  4.41  4.47  2.76   62.2   59.0   
          
             price  
          0    798  
          1   2089  
          2    990  
          3    631  
          4    579  
\end{Verbatim}
            
    \begin{Verbatim}[commandchars=\\\{\}]
{\color{incolor}In [{\color{incolor}190}]:} \PY{c+c1}{\PYZsh{} Verificando tamanho do dataframe}
          \PY{n}{train}\PY{o}{.}\PY{n}{shape}
\end{Verbatim}

\begin{Verbatim}[commandchars=\\\{\}]
{\color{outcolor}Out[{\color{outcolor}190}]:} (33940, 11)
\end{Verbatim}
            
    \begin{Verbatim}[commandchars=\\\{\}]
{\color{incolor}In [{\color{incolor}191}]:} \PY{c+c1}{\PYZsh{} Verificando informacoes especificas}
          \PY{n}{train}\PY{o}{.}\PY{n}{info}\PY{p}{(}\PY{p}{)}
\end{Verbatim}

    \begin{Verbatim}[commandchars=\\\{\}]
<class 'pandas.core.frame.DataFrame'>
RangeIndex: 33940 entries, 0 to 33939
Data columns (total 11 columns):
id         33940 non-null int64
carat      33940 non-null float64
cut        33940 non-null object
color      33940 non-null object
clarity    33940 non-null object
x          33940 non-null float64
y          33940 non-null float64
z          33940 non-null float64
depth      33940 non-null float64
table      33940 non-null float64
price      33940 non-null int64
dtypes: float64(6), int64(2), object(3)
memory usage: 2.8+ MB

    \end{Verbatim}

    \begin{Verbatim}[commandchars=\\\{\}]
{\color{incolor}In [{\color{incolor}192}]:} \PY{c+c1}{\PYZsh{} Setando o index do arquivo como index do dataframe}
          \PY{n}{train} \PY{o}{=} \PY{n}{train}\PY{o}{.}\PY{n}{set\PYZus{}index}\PY{p}{(}\PY{l+s+s1}{\PYZsq{}}\PY{l+s+s1}{id}\PY{l+s+s1}{\PYZsq{}}\PY{p}{)}
          \PY{n}{train}\PY{o}{.}\PY{n}{head}\PY{p}{(}\PY{p}{)}
\end{Verbatim}

\begin{Verbatim}[commandchars=\\\{\}]
{\color{outcolor}Out[{\color{outcolor}192}]:}        carat        cut color clarity     x     y     z  depth  table  price
          id                                                                          
          20000   0.35  Very Good     G    VVS2  4.44  4.48  2.80   62.8   58.0    798
          20001   0.70      Ideal     F     SI1  5.66  5.69  3.55   62.6   56.0   2089
          20002   0.32      Ideal     F    VVS1  4.42  4.38  2.70   61.4   56.0    990
          20003   0.30      Ideal     H    VVS2  4.32  4.35  2.67   61.7   54.2    631
          20004   0.33    Premium     I    VVS2  4.41  4.47  2.76   62.2   59.0    579
\end{Verbatim}
            
    \begin{Verbatim}[commandchars=\\\{\}]
{\color{incolor}In [{\color{incolor}193}]:} \PY{c+c1}{\PYZsh{} Verificando se existem valores nulos para o conjunto de treino}
          \PY{n}{train}\PY{o}{.}\PY{n}{isnull}\PY{p}{(}\PY{p}{)}\PY{o}{.}\PY{n}{sum}\PY{p}{(}\PY{p}{)}
\end{Verbatim}

\begin{Verbatim}[commandchars=\\\{\}]
{\color{outcolor}Out[{\color{outcolor}193}]:} carat      0
          cut        0
          color      0
          clarity    0
          x          0
          y          0
          z          0
          depth      0
          table      0
          price      0
          dtype: int64
\end{Verbatim}
            
    \begin{Verbatim}[commandchars=\\\{\}]
{\color{incolor}In [{\color{incolor}194}]:} \PY{c+c1}{\PYZsh{} Verificando os detalhes de cada caracteristica}
          \PY{n}{train}\PY{o}{.}\PY{n}{describe}\PY{p}{(}\PY{p}{)}
\end{Verbatim}

\begin{Verbatim}[commandchars=\\\{\}]
{\color{outcolor}Out[{\color{outcolor}194}]:}               carat             x             y             z         depth  \textbackslash{}
          count  33940.000000  33940.000000  33940.000000  33940.000000  33940.000000   
          mean       0.796249      5.727926      5.730563      3.535916     61.746491   
          std        0.472866      1.119282      1.120279      0.693763      1.425570   
          min        0.200000      0.000000      0.000000      0.000000     43.000000   
          25\%        0.400000      4.710000      4.720000      2.910000     61.000000   
          50\%        0.700000      5.700000      5.710000      3.520000     61.800000   
          75\%        1.040000      6.540000      6.530000      4.030000     62.500000   
          max        5.010000     10.740000     31.800000      6.980000     79.000000   
          
                        table         price  
          count  33940.000000  33940.000000  
          mean      57.467664   3920.022864  
          std        2.237116   3980.229999  
          min       44.000000    326.000000  
          25\%       56.000000    952.000000  
          50\%       57.000000   2395.000000  
          75\%       59.000000   5294.000000  
          max       95.000000  18823.000000  
\end{Verbatim}
            
    Como x, y e z são variáveis relacionadas às dimensões de cada diamante,
não faz sentido que nenhuma delas seja igual a 0. Assim, é necessário
retirar estes dados do conjunto de treino para que o modelo não seja
prejudicado.

    \begin{Verbatim}[commandchars=\\\{\}]
{\color{incolor}In [{\color{incolor}195}]:} \PY{c+c1}{\PYZsh{} Para realizar este processamento, redefine\PYZhy{}se o conjunto de treino}
          \PY{c+c1}{\PYZsh{} como todos os dados em que x, y e z sao diferentes de 0}
          \PY{n}{train} \PY{o}{=} \PY{n}{train}\PY{p}{[}\PY{p}{(}\PY{n}{train}\PY{p}{[}\PY{p}{[}\PY{l+s+s1}{\PYZsq{}}\PY{l+s+s1}{x}\PY{l+s+s1}{\PYZsq{}}\PY{p}{,}\PY{l+s+s1}{\PYZsq{}}\PY{l+s+s1}{y}\PY{l+s+s1}{\PYZsq{}}\PY{p}{,}\PY{l+s+s1}{\PYZsq{}}\PY{l+s+s1}{z}\PY{l+s+s1}{\PYZsq{}}\PY{p}{]}\PY{p}{]} \PY{o}{!=} \PY{l+m+mi}{0}\PY{p}{)}\PY{o}{.}\PY{n}{all}\PY{p}{(}\PY{n}{axis}\PY{o}{=}\PY{l+m+mi}{1}\PY{p}{)}\PY{p}{]}
          
          \PY{c+c1}{\PYZsh{} Para confirmar}
          \PY{n}{train}\PY{o}{.}\PY{n}{describe}\PY{p}{(}\PY{p}{)}
\end{Verbatim}

\begin{Verbatim}[commandchars=\\\{\}]
{\color{outcolor}Out[{\color{outcolor}195}]:}               carat             x             y             z         depth  \textbackslash{}
          count  33929.000000  33929.000000  33929.000000  33929.000000  33929.000000   
          mean       0.796061      5.728073      5.730722      3.537062     61.746754   
          std        0.472740      1.117848      1.118862      0.690948      1.425311   
          min        0.200000      3.730000      3.680000      1.070000     43.000000   
          25\%        0.400000      4.710000      4.720000      2.910000     61.000000   
          50\%        0.700000      5.690000      5.710000      3.520000     61.800000   
          75\%        1.040000      6.530000      6.530000      4.030000     62.500000   
          max        5.010000     10.740000     31.800000      6.980000     79.000000   
          
                       table         price  
          count  33929.00000  33929.000000  
          mean      57.46752   3918.401692  
          std        2.23705   3978.347387  
          min       44.00000    326.000000  
          25\%       56.00000    952.000000  
          50\%       57.00000   2394.000000  
          75\%       59.00000   5293.000000  
          max       95.00000  18823.000000  
\end{Verbatim}
            
    Como pode ser observado, agora, o conjunto de treino não possui mais x,
y e z zerados, o que se confirma pelos valores encontrados na linha de
mínimos de cada um desses atributos acima.

    \hypertarget{carregando-conjunto-de-teste}{%
\subsubsection{Carregando Conjunto de
Teste}\label{carregando-conjunto-de-teste}}

    \begin{Verbatim}[commandchars=\\\{\}]
{\color{incolor}In [{\color{incolor}196}]:} \PY{c+c1}{\PYZsh{} Carregando os dados de teste como dataframe}
          \PY{n}{test} \PY{o}{=} \PY{n}{pd}\PY{o}{.}\PY{n}{read\PYZus{}csv}\PY{p}{(}\PY{l+s+s1}{\PYZsq{}}\PY{l+s+s1}{data/test.csv}\PY{l+s+s1}{\PYZsq{}}\PY{p}{)}
          \PY{n}{test}\PY{o}{.}\PY{n}{head}\PY{p}{(}\PY{p}{)}
\end{Verbatim}

\begin{Verbatim}[commandchars=\\\{\}]
{\color{outcolor}Out[{\color{outcolor}196}]:}    id  carat        cut color clarity     x     y     z  depth  table
          0   0   1.82    Premium     G     SI1  7.75  7.68  4.84   62.7   58.0
          1   1   1.11  Very Good     H     SI1  6.63  6.65  4.11   61.9   58.0
          2   2   0.52      Ideal     D    VVS2  5.19  5.22  3.20   61.5   55.0
          3   3   1.05      Ideal     I     VS2  6.52  6.50  4.05   62.2   56.0
          4   4   0.70      Ideal     I    VVS2  5.63  5.68  3.51   62.1   58.0
\end{Verbatim}
            
    \begin{Verbatim}[commandchars=\\\{\}]
{\color{incolor}In [{\color{incolor}197}]:} \PY{n}{test}\PY{o}{.}\PY{n}{shape}
\end{Verbatim}

\begin{Verbatim}[commandchars=\\\{\}]
{\color{outcolor}Out[{\color{outcolor}197}]:} (20000, 10)
\end{Verbatim}
            
    \begin{Verbatim}[commandchars=\\\{\}]
{\color{incolor}In [{\color{incolor}198}]:} \PY{n}{test}\PY{o}{.}\PY{n}{info}\PY{p}{(}\PY{p}{)}
\end{Verbatim}

    \begin{Verbatim}[commandchars=\\\{\}]
<class 'pandas.core.frame.DataFrame'>
RangeIndex: 20000 entries, 0 to 19999
Data columns (total 10 columns):
id         20000 non-null int64
carat      20000 non-null float64
cut        20000 non-null object
color      20000 non-null object
clarity    20000 non-null object
x          20000 non-null float64
y          20000 non-null float64
z          20000 non-null float64
depth      20000 non-null float64
table      20000 non-null float64
dtypes: float64(6), int64(1), object(3)
memory usage: 1.5+ MB

    \end{Verbatim}

    \begin{Verbatim}[commandchars=\\\{\}]
{\color{incolor}In [{\color{incolor}199}]:} \PY{c+c1}{\PYZsh{} Setando o index do arquivo como index do dataframe}
          \PY{n}{test} \PY{o}{=} \PY{n}{test}\PY{o}{.}\PY{n}{set\PYZus{}index}\PY{p}{(}\PY{l+s+s1}{\PYZsq{}}\PY{l+s+s1}{id}\PY{l+s+s1}{\PYZsq{}}\PY{p}{)}
          \PY{n}{test}\PY{o}{.}\PY{n}{head}\PY{p}{(}\PY{p}{)}
\end{Verbatim}

\begin{Verbatim}[commandchars=\\\{\}]
{\color{outcolor}Out[{\color{outcolor}199}]:}     carat        cut color clarity     x     y     z  depth  table
          id                                                                
          0    1.82    Premium     G     SI1  7.75  7.68  4.84   62.7   58.0
          1    1.11  Very Good     H     SI1  6.63  6.65  4.11   61.9   58.0
          2    0.52      Ideal     D    VVS2  5.19  5.22  3.20   61.5   55.0
          3    1.05      Ideal     I     VS2  6.52  6.50  4.05   62.2   56.0
          4    0.70      Ideal     I    VVS2  5.63  5.68  3.51   62.1   58.0
\end{Verbatim}
            
    \begin{Verbatim}[commandchars=\\\{\}]
{\color{incolor}In [{\color{incolor}200}]:} \PY{c+c1}{\PYZsh{} Verificando se existem valores nulos para o conjunto de teste}
          \PY{n}{test}\PY{o}{.}\PY{n}{isnull}\PY{p}{(}\PY{p}{)}\PY{o}{.}\PY{n}{sum}\PY{p}{(}\PY{p}{)}
\end{Verbatim}

\begin{Verbatim}[commandchars=\\\{\}]
{\color{outcolor}Out[{\color{outcolor}200}]:} carat      0
          cut        0
          color      0
          clarity    0
          x          0
          y          0
          z          0
          depth      0
          table      0
          dtype: int64
\end{Verbatim}
            
    \begin{Verbatim}[commandchars=\\\{\}]
{\color{incolor}In [{\color{incolor}201}]:} \PY{c+c1}{\PYZsh{} Verificando os detalhes de cada caracteristica}
          \PY{n}{test}\PY{o}{.}\PY{n}{describe}\PY{p}{(}\PY{p}{)}
\end{Verbatim}

\begin{Verbatim}[commandchars=\\\{\}]
{\color{outcolor}Out[{\color{outcolor}201}]:}               carat             x             y             z        depth  \textbackslash{}
          count  20000.000000  20000.000000  20000.000000  20000.000000  20000.00000   
          mean       0.800809      5.736641      5.741252      3.543516     61.75435   
          std        0.475947      1.125961      1.178295      0.725497      1.44453   
          min        0.200000      0.000000      0.000000      0.000000     44.00000   
          25\%        0.400000      4.720000      4.730000      2.910000     61.00000   
          50\%        0.700000      5.700000      5.710000      3.530000     61.80000   
          75\%        1.050000      6.550000      6.540000      4.040000     62.50000   
          max        3.670000      9.860000     58.900000     31.800000     79.00000   
          
                        table  
          count  20000.000000  
          mean      57.439400  
          std        2.229972  
          min       43.000000  
          25\%       56.000000  
          50\%       57.000000  
          75\%       59.000000  
          max       79.000000  
\end{Verbatim}
            
    Agora, verifica-se a distribuição de cada um dos atributos numéricos do
dataset, verificando seus padrões e outliers.

    \hypertarget{tratamento-de-dados}{%
\subsubsection{Tratamento de Dados}\label{tratamento-de-dados}}

    \begin{Verbatim}[commandchars=\\\{\}]
{\color{incolor}In [{\color{incolor}202}]:} \PY{c+c1}{\PYZsh{} Cria\PYZhy{}se a matriz de correlacao entre os atributos numericos para visualizacao inicial}
          \PY{n}{corr\PYZus{}matrix} \PY{o}{=} \PY{n}{train}\PY{o}{.}\PY{n}{corr}\PY{p}{(}\PY{p}{)}
          
          \PY{n}{plt}\PY{o}{.}\PY{n}{subplots}\PY{p}{(}\PY{n}{figsize} \PY{o}{=} \PY{p}{(}\PY{l+m+mi}{10}\PY{p}{,} \PY{l+m+mi}{10}\PY{p}{)}\PY{p}{)}
          \PY{n}{sns}\PY{o}{.}\PY{n}{heatmap}\PY{p}{(}\PY{n}{corr\PYZus{}matrix}\PY{p}{,} \PY{n}{square}\PY{o}{=}\PY{k+kc}{True}\PY{p}{,} \PY{n}{cbar}\PY{o}{=}\PY{k+kc}{True}\PY{p}{,} \PY{n}{annot} \PY{o}{=} \PY{k+kc}{True}\PY{p}{,} \PY{n}{cmap}\PY{o}{=}\PY{l+s+s1}{\PYZsq{}}\PY{l+s+s1}{Spectral}\PY{l+s+s1}{\PYZsq{}}\PY{p}{)}
          \PY{n}{plt}\PY{o}{.}\PY{n}{show}\PY{p}{(}\PY{p}{)}
\end{Verbatim}

    \begin{center}
    \adjustimage{max size={0.9\linewidth}{0.9\paperheight}}{output_27_0.png}
    \end{center}
    { \hspace*{\fill} \\}
    
    A matriz de correlação acima possui apenas os atributos numéricos do
conjunto de dados, excluindo as características categóricas descritas
anteriormente.

A partir deste gráfico, é possível observar que \textbf{carat, x, y e z}
são os atributos de maior correlação com o preço e, fatalmente, maior
correlação entre si também, já que descrevem características
extremamente dependentes umas das outras, como peso, comprimento,
largura e profundidade.

    Define-se, então, uma função de análise, para verificar a relação de
assimetria e curtose na distribuição de cada atributo numérico e para
observar seu histograma e diagrama de dispersão em relação ao preço, e
uma função de contagem de dados até determinados limiares, configurados
manualmente de acordo com o histograma.

    \begin{Verbatim}[commandchars=\\\{\}]
{\color{incolor}In [{\color{incolor}203}]:} \PY{c+c1}{\PYZsh{} Funcao de analise de atributo}
          \PY{k}{def} \PY{n+nf}{analysis}\PY{p}{(}\PY{n}{feature}\PY{p}{,} \PY{n}{hist}\PY{o}{=}\PY{k+kc}{True}\PY{p}{)}\PY{p}{:}
              
              \PY{c+c1}{\PYZsh{} Definindo os valores de Skewness e Kurtosis para analisar}
              \PY{c+c1}{\PYZsh{} a simetria e quantidade de outliers respectivamente}
              \PY{n+nb}{print}\PY{p}{(}\PY{l+s+s1}{\PYZsq{}}\PY{l+s+s1}{Skewness: }\PY{l+s+si}{\PYZob{}\PYZcb{}}\PY{l+s+s1}{\PYZsq{}}\PY{o}{.}\PY{n}{format}\PY{p}{(}\PY{n}{train}\PY{p}{[}\PY{n}{feature}\PY{p}{]}\PY{o}{.}\PY{n}{skew}\PY{p}{(}\PY{p}{)}\PY{p}{)}\PY{p}{)}
              \PY{n+nb}{print}\PY{p}{(}\PY{l+s+s1}{\PYZsq{}}\PY{l+s+s1}{Kurtosis: }\PY{l+s+si}{\PYZob{}\PYZcb{}}\PY{l+s+s1}{\PYZsq{}}\PY{o}{.}\PY{n}{format}\PY{p}{(}\PY{n}{train}\PY{p}{[}\PY{n}{feature}\PY{p}{]}\PY{o}{.}\PY{n}{kurt}\PY{p}{(}\PY{p}{)}\PY{p}{)}\PY{p}{)}
          
              \PY{k}{if} \PY{n}{hist}\PY{p}{:}
                  \PY{c+c1}{\PYZsh{} Plotando o histograma}
                  \PY{n}{plt}\PY{o}{.}\PY{n}{figure}\PY{p}{(}\PY{n}{figsize}\PY{o}{=}\PY{p}{(}\PY{l+m+mi}{20}\PY{p}{,}\PY{l+m+mi}{10}\PY{p}{)}\PY{p}{)}
                  \PY{n}{train}\PY{p}{[}\PY{n}{feature}\PY{p}{]}\PY{o}{.}\PY{n}{hist}\PY{p}{(}\PY{n}{bins} \PY{o}{=} \PY{l+m+mi}{500}\PY{p}{)}
                  \PY{n}{plt}\PY{o}{.}\PY{n}{show}\PY{p}{(}\PY{p}{)}
              
              \PY{k}{if} \PY{n}{feature} \PY{o}{!=} \PY{l+s+s1}{\PYZsq{}}\PY{l+s+s1}{price}\PY{l+s+s1}{\PYZsq{}}\PY{p}{:}
                  \PY{c+c1}{\PYZsh{} Plotando o diagrama de dispersão}
                  \PY{n}{plt}\PY{o}{.}\PY{n}{figure}\PY{p}{(}\PY{n}{figsize}\PY{o}{=}\PY{p}{(}\PY{l+m+mi}{20}\PY{p}{,}\PY{l+m+mi}{10}\PY{p}{)}\PY{p}{)}
                  \PY{n}{train}\PY{o}{.}\PY{n}{plot}\PY{o}{.}\PY{n}{scatter}\PY{p}{(}\PY{n}{x} \PY{o}{=} \PY{n}{feature}\PY{p}{,} \PY{n}{y} \PY{o}{=} \PY{l+s+s1}{\PYZsq{}}\PY{l+s+s1}{price}\PY{l+s+s1}{\PYZsq{}}\PY{p}{)}
                  \PY{n}{plt}\PY{o}{.}\PY{n}{show}\PY{p}{(}\PY{p}{)}
                  
                  
          \PY{c+c1}{\PYZsh{} Funcao que checa a contagem para cada limiar    }
          \PY{k}{def} \PY{n+nf}{count\PYZus{}limit}\PY{p}{(}\PY{n}{feature}\PY{p}{,} \PY{n}{inf\PYZus{}limit}\PY{p}{,} \PY{n}{sup\PYZus{}limit}\PY{p}{,} \PY{n}{hop}\PY{p}{)}\PY{p}{:}
              
              \PY{n}{n} \PY{o}{=} \PY{n+nb}{int}\PY{p}{(}\PY{p}{(}\PY{n}{sup\PYZus{}limit} \PY{o}{\PYZhy{}} \PY{n}{inf\PYZus{}limit}\PY{p}{)} \PY{o}{/} \PY{n}{hop}\PY{p}{)}
              \PY{n}{p} \PY{o}{=} \PY{n}{np}\PY{o}{.}\PY{n}{zeros}\PY{p}{(}\PY{p}{(}\PY{p}{(}\PY{n}{n} \PY{o}{+} \PY{l+m+mi}{1}\PY{p}{)}\PY{p}{,} \PY{l+m+mi}{2}\PY{p}{)}\PY{p}{)}
              
              \PY{k}{for} \PY{n}{i} \PY{o+ow}{in} \PY{n+nb}{range}\PY{p}{(}\PY{n}{n} \PY{o}{+} \PY{l+m+mi}{1}\PY{p}{)}\PY{p}{:}
                  \PY{n}{p}\PY{p}{[}\PY{n}{i}\PY{p}{]}\PY{p}{[}\PY{l+m+mi}{0}\PY{p}{]} \PY{o}{=} \PY{n}{train}\PY{p}{[}\PY{n}{train}\PY{p}{[}\PY{n}{feature}\PY{p}{]} \PY{o}{\PYZlt{}} \PY{p}{(}\PY{n}{inf\PYZus{}limit} \PY{o}{+} \PY{p}{(}\PY{n}{hop} \PY{o}{*} \PY{n}{i}\PY{p}{)}\PY{p}{)}\PY{p}{]}\PY{p}{[}\PY{n}{feature}\PY{p}{]}\PY{o}{.}\PY{n}{count}\PY{p}{(}\PY{p}{)}
                  \PY{n}{p}\PY{p}{[}\PY{n}{i}\PY{p}{]}\PY{p}{[}\PY{l+m+mi}{1}\PY{p}{]} \PY{o}{=} \PY{n}{np}\PY{o}{.}\PY{n}{round}\PY{p}{(}\PY{p}{(}\PY{n}{p}\PY{p}{[}\PY{n}{i}\PY{p}{]}\PY{p}{[}\PY{l+m+mi}{0}\PY{p}{]} \PY{o}{/} \PY{n}{train}\PY{p}{[}\PY{n}{train}\PY{p}{[}\PY{n}{feature}\PY{p}{]} \PY{o}{\PYZlt{}} \PY{n}{sup\PYZus{}limit}\PY{p}{]}\PY{p}{[}\PY{n}{feature}\PY{p}{]}\PY{o}{.}\PY{n}{count}\PY{p}{(}\PY{p}{)}\PY{p}{)} \PY{o}{*} \PY{l+m+mi}{100}\PY{p}{,} \PY{l+m+mi}{2}\PY{p}{)}
                  \PY{n+nb}{print}\PY{p}{(}\PY{l+s+s1}{\PYZsq{}}\PY{l+s+s1}{Quantidade de pontos abaixo de }\PY{l+s+si}{\PYZob{}\PYZcb{}}\PY{l+s+s1}{ :}\PY{l+s+s1}{\PYZsq{}}\PY{o}{.}\PY{n}{format}\PY{p}{(}\PY{n}{inf\PYZus{}limit} \PY{o}{+} \PY{p}{(}\PY{n}{hop} \PY{o}{*} \PY{n}{i}\PY{p}{)}\PY{p}{)}\PY{p}{,} \PY{n}{p}\PY{p}{[}\PY{n}{i}\PY{p}{]}\PY{p}{[}\PY{l+m+mi}{0}\PY{p}{]}\PY{p}{,} \PYZbs{}
                        \PY{l+s+s1}{\PYZsq{}}\PY{l+s+s1}{Porcentagem: }\PY{l+s+si}{\PYZob{}\PYZcb{}}\PY{l+s+s1}{ }\PY{l+s+s1}{\PYZpc{}}\PY{l+s+s1}{\PYZsq{}}\PY{o}{.}\PY{n}{format}\PY{p}{(}\PY{n}{p}\PY{p}{[}\PY{n}{i}\PY{p}{]}\PY{p}{[}\PY{l+m+mi}{1}\PY{p}{]}\PY{p}{)}\PY{p}{)}
\end{Verbatim}

    \hypertarget{preuxe7o}{%
\paragraph{Preço}\label{preuxe7o}}

    \begin{Verbatim}[commandchars=\\\{\}]
{\color{incolor}In [{\color{incolor}204}]:} \PY{n}{train}\PY{p}{[}\PY{l+s+s1}{\PYZsq{}}\PY{l+s+s1}{price}\PY{l+s+s1}{\PYZsq{}}\PY{p}{]}\PY{o}{.}\PY{n}{describe}\PY{p}{(}\PY{p}{)}
\end{Verbatim}

\begin{Verbatim}[commandchars=\\\{\}]
{\color{outcolor}Out[{\color{outcolor}204}]:} count    33929.000000
          mean      3918.401692
          std       3978.347387
          min        326.000000
          25\%        952.000000
          50\%       2394.000000
          75\%       5293.000000
          max      18823.000000
          Name: price, dtype: float64
\end{Verbatim}
            
    \begin{Verbatim}[commandchars=\\\{\}]
{\color{incolor}In [{\color{incolor}205}]:} \PY{n}{analysis}\PY{p}{(}\PY{l+s+s1}{\PYZsq{}}\PY{l+s+s1}{price}\PY{l+s+s1}{\PYZsq{}}\PY{p}{)}
\end{Verbatim}

    \begin{Verbatim}[commandchars=\\\{\}]
Skewness: 1.6292963158992595
Kurtosis: 2.224956771069901

    \end{Verbatim}

    \begin{center}
    \adjustimage{max size={0.9\linewidth}{0.9\paperheight}}{output_33_1.png}
    \end{center}
    { \hspace*{\fill} \\}
    
    \begin{Verbatim}[commandchars=\\\{\}]
{\color{incolor}In [{\color{incolor}206}]:} \PY{n}{count\PYZus{}limit}\PY{p}{(}\PY{l+s+s1}{\PYZsq{}}\PY{l+s+s1}{price}\PY{l+s+s1}{\PYZsq{}}\PY{p}{,} \PY{l+m+mi}{2500}\PY{p}{,} \PY{l+m+mi}{20000}\PY{p}{,} \PY{l+m+mi}{2500}\PY{p}{)}
\end{Verbatim}

    \begin{Verbatim}[commandchars=\\\{\}]
Quantidade de pontos abaixo de 2500 : 17381.0 Porcentagem: 51.23 \%
Quantidade de pontos abaixo de 5000 : 24717.0 Porcentagem: 72.85 \%
Quantidade de pontos abaixo de 7500 : 28734.0 Porcentagem: 84.69 \%
Quantidade de pontos abaixo de 10000 : 30664.0 Porcentagem: 90.38 \%
Quantidade de pontos abaixo de 12500 : 31976.0 Porcentagem: 94.24 \%
Quantidade de pontos abaixo de 15000 : 32881.0 Porcentagem: 96.91 \%
Quantidade de pontos abaixo de 17500 : 33630.0 Porcentagem: 99.12 \%
Quantidade de pontos abaixo de 20000 : 33929.0 Porcentagem: 100.0 \%

    \end{Verbatim}

    \hypertarget{removendo-os-outliers}{%
\subparagraph{Removendo os Outliers}\label{removendo-os-outliers}}

    Como o preço é o alvo a ser estimado pelo modelo, não faz sentido
remover nenhum valor específico.

    \hypertarget{carat}{%
\paragraph{Carat}\label{carat}}

    \begin{Verbatim}[commandchars=\\\{\}]
{\color{incolor}In [{\color{incolor}207}]:} \PY{n}{train}\PY{p}{[}\PY{l+s+s1}{\PYZsq{}}\PY{l+s+s1}{carat}\PY{l+s+s1}{\PYZsq{}}\PY{p}{]}\PY{o}{.}\PY{n}{describe}\PY{p}{(}\PY{p}{)}
\end{Verbatim}

\begin{Verbatim}[commandchars=\\\{\}]
{\color{outcolor}Out[{\color{outcolor}207}]:} count    33929.000000
          mean         0.796061
          std          0.472740
          min          0.200000
          25\%          0.400000
          50\%          0.700000
          75\%          1.040000
          max          5.010000
          Name: carat, dtype: float64
\end{Verbatim}
            
    \begin{Verbatim}[commandchars=\\\{\}]
{\color{incolor}In [{\color{incolor}208}]:} \PY{n}{analysis}\PY{p}{(}\PY{l+s+s1}{\PYZsq{}}\PY{l+s+s1}{carat}\PY{l+s+s1}{\PYZsq{}}\PY{p}{)}
\end{Verbatim}

    \begin{Verbatim}[commandchars=\\\{\}]
Skewness: 1.1315062541047431
Kurtosis: 1.4017522300920007

    \end{Verbatim}

    \begin{center}
    \adjustimage{max size={0.9\linewidth}{0.9\paperheight}}{output_39_1.png}
    \end{center}
    { \hspace*{\fill} \\}
    
    
    \begin{verbatim}
<Figure size 1440x720 with 0 Axes>
    \end{verbatim}

    
    \begin{center}
    \adjustimage{max size={0.9\linewidth}{0.9\paperheight}}{output_39_3.png}
    \end{center}
    { \hspace*{\fill} \\}
    
    \begin{Verbatim}[commandchars=\\\{\}]
{\color{incolor}In [{\color{incolor}209}]:} \PY{n}{count\PYZus{}limit}\PY{p}{(}\PY{l+s+s1}{\PYZsq{}}\PY{l+s+s1}{carat}\PY{l+s+s1}{\PYZsq{}}\PY{p}{,} \PY{l+m+mf}{0.5}\PY{p}{,} \PY{l+m+mi}{3}\PY{p}{,} \PY{l+m+mf}{0.5}\PY{p}{)}
\end{Verbatim}

    \begin{Verbatim}[commandchars=\\\{\}]
Quantidade de pontos abaixo de 0.5 : 11121.0 Porcentagem: 32.8 \%
Quantidade de pontos abaixo de 1.0 : 21961.0 Porcentagem: 64.77 \%
Quantidade de pontos abaixo de 1.5 : 30074.0 Porcentagem: 88.7 \%
Quantidade de pontos abaixo de 2.0 : 32589.0 Porcentagem: 96.12 \%
Quantidade de pontos abaixo de 2.5 : 33847.0 Porcentagem: 99.83 \%
Quantidade de pontos abaixo de 3.0 : 33905.0 Porcentagem: 100.0 \%

    \end{Verbatim}

    \hypertarget{removendo-os-outliers}{%
\subparagraph{Removendo os Outliers}\label{removendo-os-outliers}}

    De acordo com a observação do gráfico acima e da quantidade de dados
acumulados, os outliers poderiam ser considerados com \textbf{carat}
abaixo de 2.4, por exemplo. Além disso, é importante observar o
comportamento espaçado do atributo, assumindo conjuntos de valores a
partir de determinadas ``linhas bem definidas''.\\
O modelo foi testado considerando a remoção deste dados, porém,
apresentou melhores resultados considerando todos os casos, por isso, o
código abaixo se encontra comentado.

    \begin{Verbatim}[commandchars=\\\{\}]
{\color{incolor}In [{\color{incolor}210}]:} \PY{c+c1}{\PYZsh{} train = train[train[\PYZsq{}carat\PYZsq{}] \PYZlt{} 2.4]}
          
          \PY{c+c1}{\PYZsh{} analysis(\PYZsq{}carat\PYZsq{}, hist=False)}
\end{Verbatim}

    \hypertarget{x}{%
\paragraph{x}\label{x}}

    \begin{Verbatim}[commandchars=\\\{\}]
{\color{incolor}In [{\color{incolor}211}]:} \PY{n}{train}\PY{p}{[}\PY{l+s+s1}{\PYZsq{}}\PY{l+s+s1}{x}\PY{l+s+s1}{\PYZsq{}}\PY{p}{]}\PY{o}{.}\PY{n}{describe}\PY{p}{(}\PY{p}{)}
\end{Verbatim}

\begin{Verbatim}[commandchars=\\\{\}]
{\color{outcolor}Out[{\color{outcolor}211}]:} count    33929.000000
          mean         5.728073
          std          1.117848
          min          3.730000
          25\%          4.710000
          50\%          5.690000
          75\%          6.530000
          max         10.740000
          Name: x, dtype: float64
\end{Verbatim}
            
    \begin{Verbatim}[commandchars=\\\{\}]
{\color{incolor}In [{\color{incolor}212}]:} \PY{n}{analysis}\PY{p}{(}\PY{l+s+s1}{\PYZsq{}}\PY{l+s+s1}{x}\PY{l+s+s1}{\PYZsq{}}\PY{p}{)}
\end{Verbatim}

    \begin{Verbatim}[commandchars=\\\{\}]
Skewness: 0.3974954116254315
Kurtosis: -0.694819285807756

    \end{Verbatim}

    \begin{center}
    \adjustimage{max size={0.9\linewidth}{0.9\paperheight}}{output_46_1.png}
    \end{center}
    { \hspace*{\fill} \\}
    
    
    \begin{verbatim}
<Figure size 1440x720 with 0 Axes>
    \end{verbatim}

    
    \begin{center}
    \adjustimage{max size={0.9\linewidth}{0.9\paperheight}}{output_46_3.png}
    \end{center}
    { \hspace*{\fill} \\}
    
    \begin{Verbatim}[commandchars=\\\{\}]
{\color{incolor}In [{\color{incolor}213}]:} \PY{n}{count\PYZus{}limit}\PY{p}{(}\PY{l+s+s1}{\PYZsq{}}\PY{l+s+s1}{x}\PY{l+s+s1}{\PYZsq{}}\PY{p}{,} \PY{l+m+mi}{4}\PY{p}{,} \PY{l+m+mi}{9}\PY{p}{,} \PY{l+m+mi}{1}\PY{p}{)}
\end{Verbatim}

    \begin{Verbatim}[commandchars=\\\{\}]
Quantidade de pontos abaixo de 4 : 292.0 Porcentagem: 0.86 \%
Quantidade de pontos abaixo de 5 : 11051.0 Porcentagem: 32.59 \%
Quantidade de pontos abaixo de 6 : 19886.0 Porcentagem: 58.65 \%
Quantidade de pontos abaixo de 7 : 29416.0 Porcentagem: 86.76 \%
Quantidade de pontos abaixo de 8 : 32750.0 Porcentagem: 96.6 \%
Quantidade de pontos abaixo de 9 : 33904.0 Porcentagem: 100.0 \%

    \end{Verbatim}

    \hypertarget{removendo-os-outliers}{%
\subparagraph{Removendo os Outliers}\label{removendo-os-outliers}}

    De acordo com a observação do gráfico acima e da quantidade de dados
acumulados, os outliers poderiam ser considerados com \textbf{x} abaixo
de 9, por exemplo.\\
O modelo foi testado considerando a remoção deste dados, porém,
apresentou melhores resultados considerando todos os casos, por isso, o
código abaixo se encontra comentado.

    \begin{Verbatim}[commandchars=\\\{\}]
{\color{incolor}In [{\color{incolor}214}]:} \PY{c+c1}{\PYZsh{} train = train[train[\PYZsq{}x\PYZsq{}] \PYZlt{} 9]}
          
          \PY{c+c1}{\PYZsh{} analysis(\PYZsq{}x\PYZsq{}, hist=False)}
\end{Verbatim}

    \hypertarget{y}{%
\paragraph{y}\label{y}}

    \begin{Verbatim}[commandchars=\\\{\}]
{\color{incolor}In [{\color{incolor}215}]:} \PY{n}{train}\PY{p}{[}\PY{l+s+s1}{\PYZsq{}}\PY{l+s+s1}{y}\PY{l+s+s1}{\PYZsq{}}\PY{p}{]}\PY{o}{.}\PY{n}{describe}\PY{p}{(}\PY{p}{)}
\end{Verbatim}

\begin{Verbatim}[commandchars=\\\{\}]
{\color{outcolor}Out[{\color{outcolor}215}]:} count    33929.000000
          mean         5.730722
          std          1.118862
          min          3.680000
          25\%          4.720000
          50\%          5.710000
          75\%          6.530000
          max         31.800000
          Name: y, dtype: float64
\end{Verbatim}
            
    \begin{Verbatim}[commandchars=\\\{\}]
{\color{incolor}In [{\color{incolor}216}]:} \PY{n}{analysis}\PY{p}{(}\PY{l+s+s1}{\PYZsq{}}\PY{l+s+s1}{y}\PY{l+s+s1}{\PYZsq{}}\PY{p}{)}
\end{Verbatim}

    \begin{Verbatim}[commandchars=\\\{\}]
Skewness: 0.7538922955405701
Kurtosis: 7.907805795324959

    \end{Verbatim}

    \begin{center}
    \adjustimage{max size={0.9\linewidth}{0.9\paperheight}}{output_53_1.png}
    \end{center}
    { \hspace*{\fill} \\}
    
    
    \begin{verbatim}
<Figure size 1440x720 with 0 Axes>
    \end{verbatim}

    
    \begin{center}
    \adjustimage{max size={0.9\linewidth}{0.9\paperheight}}{output_53_3.png}
    \end{center}
    { \hspace*{\fill} \\}
    
    \begin{Verbatim}[commandchars=\\\{\}]
{\color{incolor}In [{\color{incolor}217}]:} \PY{n}{count\PYZus{}limit}\PY{p}{(}\PY{l+s+s1}{\PYZsq{}}\PY{l+s+s1}{y}\PY{l+s+s1}{\PYZsq{}}\PY{p}{,} \PY{l+m+mi}{5}\PY{p}{,} \PY{l+m+mi}{35}\PY{p}{,} \PY{l+m+mi}{5}\PY{p}{)}
\end{Verbatim}

    \begin{Verbatim}[commandchars=\\\{\}]
Quantidade de pontos abaixo de 5 : 11044.0 Porcentagem: 32.55 \%
Quantidade de pontos abaixo de 10 : 33925.0 Porcentagem: 99.99 \%
Quantidade de pontos abaixo de 15 : 33928.0 Porcentagem: 100.0 \%
Quantidade de pontos abaixo de 20 : 33928.0 Porcentagem: 100.0 \%
Quantidade de pontos abaixo de 25 : 33928.0 Porcentagem: 100.0 \%
Quantidade de pontos abaixo de 30 : 33928.0 Porcentagem: 100.0 \%
Quantidade de pontos abaixo de 35 : 33929.0 Porcentagem: 100.0 \%

    \end{Verbatim}

    \hypertarget{removendo-os-outliers}{%
\subparagraph{Removendo os Outliers}\label{removendo-os-outliers}}

    De acordo com a observação do gráfico acima e da quantidade de dados
acumulados, os outliers poderiam ser considerados com \textbf{y} abaixo
de 10, por exemplo.\\
O modelo foi testado considerando a remoção deste dados, porém,
apresentou melhores resultados considerando todos os casos, por isso, o
código abaixo se encontra comentado.

    \begin{Verbatim}[commandchars=\\\{\}]
{\color{incolor}In [{\color{incolor}218}]:} \PY{c+c1}{\PYZsh{} train = train[train[\PYZsq{}y\PYZsq{}] \PYZlt{} 10]}
          
          \PY{c+c1}{\PYZsh{} analysis(\PYZsq{}y\PYZsq{})}
\end{Verbatim}

    \hypertarget{z}{%
\paragraph{z}\label{z}}

    \begin{Verbatim}[commandchars=\\\{\}]
{\color{incolor}In [{\color{incolor}219}]:} \PY{n}{train}\PY{p}{[}\PY{l+s+s1}{\PYZsq{}}\PY{l+s+s1}{z}\PY{l+s+s1}{\PYZsq{}}\PY{p}{]}\PY{o}{.}\PY{n}{describe}\PY{p}{(}\PY{p}{)}
\end{Verbatim}

\begin{Verbatim}[commandchars=\\\{\}]
{\color{outcolor}Out[{\color{outcolor}219}]:} count    33929.000000
          mean         3.537062
          std          0.690948
          min          1.070000
          25\%          2.910000
          50\%          3.520000
          75\%          4.030000
          max          6.980000
          Name: z, dtype: float64
\end{Verbatim}
            
    \begin{Verbatim}[commandchars=\\\{\}]
{\color{incolor}In [{\color{incolor}220}]:} \PY{n}{analysis}\PY{p}{(}\PY{l+s+s1}{\PYZsq{}}\PY{l+s+s1}{z}\PY{l+s+s1}{\PYZsq{}}\PY{p}{)}
\end{Verbatim}

    \begin{Verbatim}[commandchars=\\\{\}]
Skewness: 0.39476084546698736
Kurtosis: -0.6716881124170859

    \end{Verbatim}

    \begin{center}
    \adjustimage{max size={0.9\linewidth}{0.9\paperheight}}{output_60_1.png}
    \end{center}
    { \hspace*{\fill} \\}
    
    
    \begin{verbatim}
<Figure size 1440x720 with 0 Axes>
    \end{verbatim}

    
    \begin{center}
    \adjustimage{max size={0.9\linewidth}{0.9\paperheight}}{output_60_3.png}
    \end{center}
    { \hspace*{\fill} \\}
    
    \begin{Verbatim}[commandchars=\\\{\}]
{\color{incolor}In [{\color{incolor}221}]:} \PY{n}{count\PYZus{}limit}\PY{p}{(}\PY{l+s+s1}{\PYZsq{}}\PY{l+s+s1}{z}\PY{l+s+s1}{\PYZsq{}}\PY{p}{,} \PY{l+m+mi}{1}\PY{p}{,} \PY{l+m+mi}{6}\PY{p}{,} \PY{l+m+mf}{0.5}\PY{p}{)}
\end{Verbatim}

    \begin{Verbatim}[commandchars=\\\{\}]
Quantidade de pontos abaixo de 1.0 : 0.0 Porcentagem: 0.0 \%
Quantidade de pontos abaixo de 1.5 : 2.0 Porcentagem: 0.01 \%
Quantidade de pontos abaixo de 2.0 : 3.0 Porcentagem: 0.01 \%
Quantidade de pontos abaixo de 2.5 : 418.0 Porcentagem: 1.23 \%
Quantidade de pontos abaixo de 3.0 : 10410.0 Porcentagem: 30.69 \%
Quantidade de pontos abaixo de 3.5 : 16446.0 Porcentagem: 48.48 \%
Quantidade de pontos abaixo de 4.0 : 24190.0 Porcentagem: 71.31 \%
Quantidade de pontos abaixo de 4.5 : 30397.0 Porcentagem: 89.61 \%
Quantidade de pontos abaixo de 5.0 : 32955.0 Porcentagem: 97.16 \%
Quantidade de pontos abaixo de 5.5 : 33891.0 Porcentagem: 99.91 \%
Quantidade de pontos abaixo de 6.0 : 33920.0 Porcentagem: 100.0 \%

    \end{Verbatim}

    \hypertarget{removendo-os-outliers}{%
\subparagraph{Removendo os Outliers}\label{removendo-os-outliers}}

    De acordo com a observação do gráfico acima e da quantidade de dados
acumulados, os outliers poderiam ser considerados com \textbf{z} acima
de 2.2 e abaixo de 5.4, por exemplo.\\
O modelo foi testado considerando a remoção deste dados, porém,
apresentou melhores resultados considerando todos os casos, por isso, o
código abaixo se encontra comentado.

    \begin{Verbatim}[commandchars=\\\{\}]
{\color{incolor}In [{\color{incolor}222}]:} \PY{c+c1}{\PYZsh{} train = train[train[\PYZsq{}z\PYZsq{}] \PYZgt{} 2.2]}
          \PY{c+c1}{\PYZsh{} train = train[train[\PYZsq{}z\PYZsq{}] \PYZlt{} 5.4]}
          
          \PY{c+c1}{\PYZsh{} analysis(\PYZsq{}z\PYZsq{})}
\end{Verbatim}

    \hypertarget{depth}{%
\paragraph{depth}\label{depth}}

    \begin{Verbatim}[commandchars=\\\{\}]
{\color{incolor}In [{\color{incolor}223}]:} \PY{n}{train}\PY{p}{[}\PY{l+s+s1}{\PYZsq{}}\PY{l+s+s1}{depth}\PY{l+s+s1}{\PYZsq{}}\PY{p}{]}\PY{o}{.}\PY{n}{describe}\PY{p}{(}\PY{p}{)}
\end{Verbatim}

\begin{Verbatim}[commandchars=\\\{\}]
{\color{outcolor}Out[{\color{outcolor}223}]:} count    33929.000000
          mean        61.746754
          std          1.425311
          min         43.000000
          25\%         61.000000
          50\%         61.800000
          75\%         62.500000
          max         79.000000
          Name: depth, dtype: float64
\end{Verbatim}
            
    \begin{Verbatim}[commandchars=\\\{\}]
{\color{incolor}In [{\color{incolor}224}]:} \PY{n}{analysis}\PY{p}{(}\PY{l+s+s1}{\PYZsq{}}\PY{l+s+s1}{depth}\PY{l+s+s1}{\PYZsq{}}\PY{p}{)}
\end{Verbatim}

    \begin{Verbatim}[commandchars=\\\{\}]
Skewness: -0.13595475750235045
Kurtosis: 5.467708844561626

    \end{Verbatim}

    \begin{center}
    \adjustimage{max size={0.9\linewidth}{0.9\paperheight}}{output_67_1.png}
    \end{center}
    { \hspace*{\fill} \\}
    
    
    \begin{verbatim}
<Figure size 1440x720 with 0 Axes>
    \end{verbatim}

    
    \begin{center}
    \adjustimage{max size={0.9\linewidth}{0.9\paperheight}}{output_67_3.png}
    \end{center}
    { \hspace*{\fill} \\}
    
    \begin{Verbatim}[commandchars=\\\{\}]
{\color{incolor}In [{\color{incolor}225}]:} \PY{n}{count\PYZus{}limit}\PY{p}{(}\PY{l+s+s1}{\PYZsq{}}\PY{l+s+s1}{depth}\PY{l+s+s1}{\PYZsq{}}\PY{p}{,} \PY{l+m+mi}{45}\PY{p}{,} \PY{l+m+mi}{80}\PY{p}{,} \PY{l+m+mi}{5}\PY{p}{)}
\end{Verbatim}

    \begin{Verbatim}[commandchars=\\\{\}]
Quantidade de pontos abaixo de 45 : 2.0 Porcentagem: 0.01 \%
Quantidade de pontos abaixo de 50 : 2.0 Porcentagem: 0.01 \%
Quantidade de pontos abaixo de 55 : 13.0 Porcentagem: 0.04 \%
Quantidade de pontos abaixo de 60 : 3216.0 Porcentagem: 9.48 \%
Quantidade de pontos abaixo de 65 : 33387.0 Porcentagem: 98.4 \%
Quantidade de pontos abaixo de 70 : 33917.0 Porcentagem: 99.96 \%
Quantidade de pontos abaixo de 75 : 33928.0 Porcentagem: 100.0 \%
Quantidade de pontos abaixo de 80 : 33929.0 Porcentagem: 100.0 \%

    \end{Verbatim}

    \hypertarget{removendo-os-outliers}{%
\subparagraph{Removendo os Outliers}\label{removendo-os-outliers}}

    De acordo com a observação do gráfico acima e da quantidade de dados
acumulados, os outliers poderiam ser considerados com \textbf{depth}
acima de 56 e abaixo de 67, por exemplo.\\
O modelo foi testado considerando a remoção deste dados, porém,
apresentou melhores resultados considerando todos os casos, por isso, o
código abaixo se encontra comentado.

    \begin{Verbatim}[commandchars=\\\{\}]
{\color{incolor}In [{\color{incolor}226}]:} \PY{c+c1}{\PYZsh{} train = train[train[\PYZsq{}depth\PYZsq{}] \PYZgt{} 56]}
          \PY{c+c1}{\PYZsh{} train = train[train[\PYZsq{}depth\PYZsq{}] \PYZlt{} 67]}
          
          \PY{c+c1}{\PYZsh{} analysis(\PYZsq{}depth\PYZsq{})}
\end{Verbatim}

    \hypertarget{table}{%
\paragraph{table}\label{table}}

    \begin{Verbatim}[commandchars=\\\{\}]
{\color{incolor}In [{\color{incolor}227}]:} \PY{n}{train}\PY{p}{[}\PY{l+s+s1}{\PYZsq{}}\PY{l+s+s1}{table}\PY{l+s+s1}{\PYZsq{}}\PY{p}{]}\PY{o}{.}\PY{n}{describe}\PY{p}{(}\PY{p}{)}
\end{Verbatim}

\begin{Verbatim}[commandchars=\\\{\}]
{\color{outcolor}Out[{\color{outcolor}227}]:} count    33929.00000
          mean        57.46752
          std          2.23705
          min         44.00000
          25\%         56.00000
          50\%         57.00000
          75\%         59.00000
          max         95.00000
          Name: table, dtype: float64
\end{Verbatim}
            
    \begin{Verbatim}[commandchars=\\\{\}]
{\color{incolor}In [{\color{incolor}228}]:} \PY{n}{analysis}\PY{p}{(}\PY{l+s+s1}{\PYZsq{}}\PY{l+s+s1}{table}\PY{l+s+s1}{\PYZsq{}}\PY{p}{)}
\end{Verbatim}

    \begin{Verbatim}[commandchars=\\\{\}]
Skewness: 0.8158204615629009
Kurtosis: 3.4298697674709766

    \end{Verbatim}

    \begin{center}
    \adjustimage{max size={0.9\linewidth}{0.9\paperheight}}{output_74_1.png}
    \end{center}
    { \hspace*{\fill} \\}
    
    
    \begin{verbatim}
<Figure size 1440x720 with 0 Axes>
    \end{verbatim}

    
    \begin{center}
    \adjustimage{max size={0.9\linewidth}{0.9\paperheight}}{output_74_3.png}
    \end{center}
    { \hspace*{\fill} \\}
    
    \begin{Verbatim}[commandchars=\\\{\}]
{\color{incolor}In [{\color{incolor}229}]:} \PY{n}{count\PYZus{}limit}\PY{p}{(}\PY{l+s+s1}{\PYZsq{}}\PY{l+s+s1}{table}\PY{l+s+s1}{\PYZsq{}}\PY{p}{,} \PY{l+m+mi}{45}\PY{p}{,} \PY{l+m+mi}{75}\PY{p}{,} \PY{l+m+mi}{5}\PY{p}{)}
\end{Verbatim}

    \begin{Verbatim}[commandchars=\\\{\}]
Quantidade de pontos abaixo de 45 : 1.0 Porcentagem: 0.0 \%
Quantidade de pontos abaixo de 50 : 2.0 Porcentagem: 0.01 \%
Quantidade de pontos abaixo de 55 : 2236.0 Porcentagem: 6.59 \%
Quantidade de pontos abaixo de 60 : 28131.0 Porcentagem: 82.92 \%
Quantidade de pontos abaixo de 65 : 33739.0 Porcentagem: 99.45 \%
Quantidade de pontos abaixo de 70 : 33917.0 Porcentagem: 99.97 \%
Quantidade de pontos abaixo de 75 : 33927.0 Porcentagem: 100.0 \%

    \end{Verbatim}

    \hypertarget{removendo-os-outliers}{%
\subparagraph{Removendo os Outliers}\label{removendo-os-outliers}}

    De acordo com a observação do gráfico acima e da quantidade de dados
acumulados, os outliers poderiam ser considerados com \textbf{table}
acima de 56 e abaixo de 67, por exemplo.\\
O modelo foi testado considerando a remoção deste dados, porém,
apresentou melhores resultados considerando todos os casos, por isso, o
código abaixo se encontra comentado.

    \begin{Verbatim}[commandchars=\\\{\}]
{\color{incolor}In [{\color{incolor}230}]:} \PY{c+c1}{\PYZsh{} train = train[train[\PYZsq{}table\PYZsq{}] \PYZgt{} 56]}
          \PY{c+c1}{\PYZsh{} train = train[train[\PYZsq{}table\PYZsq{}] \PYZlt{} 67]}
          
          \PY{c+c1}{\PYZsh{} analysis(\PYZsq{}table\PYZsq{})}
\end{Verbatim}

    \hypertarget{outras-observauxe7uxf5es}{%
\subsubsection{Outras Observações}\label{outras-observauxe7uxf5es}}

    Para visualizar todos os atributos numéricos e as relações entre eles,
plotam-se vários gráficos de dispersão para evidenciar as dependências
possíveis e, novamente, a matriz de correlações, para analisar a remoção
de outliers caso ela seja aplicada.

    \begin{Verbatim}[commandchars=\\\{\}]
{\color{incolor}In [{\color{incolor}231}]:} \PY{c+c1}{\PYZsh{} Criando os graficos de dispersao para visualizacao geral}
          
          \PY{n}{sns}\PY{o}{.}\PY{n}{pairplot}\PY{p}{(}\PY{n}{train}\PY{p}{)}
\end{Verbatim}

\begin{Verbatim}[commandchars=\\\{\}]
{\color{outcolor}Out[{\color{outcolor}231}]:} <seaborn.axisgrid.PairGrid at 0x17c9d92f6a0>
\end{Verbatim}
            
    \begin{center}
    \adjustimage{max size={0.9\linewidth}{0.9\paperheight}}{output_81_1.png}
    \end{center}
    { \hspace*{\fill} \\}
    
    \begin{Verbatim}[commandchars=\\\{\}]
{\color{incolor}In [{\color{incolor}232}]:} \PY{c+c1}{\PYZsh{} Cria a matriz de correlacao entre os atributos numericos}
          \PY{n}{corr\PYZus{}matrix} \PY{o}{=} \PY{n}{train}\PY{o}{.}\PY{n}{corr}\PY{p}{(}\PY{p}{)}
          
          \PY{n}{plt}\PY{o}{.}\PY{n}{subplots}\PY{p}{(}\PY{n}{figsize} \PY{o}{=} \PY{p}{(}\PY{l+m+mi}{10}\PY{p}{,} \PY{l+m+mi}{10}\PY{p}{)}\PY{p}{)}
          \PY{n}{sns}\PY{o}{.}\PY{n}{heatmap}\PY{p}{(}\PY{n}{corr\PYZus{}matrix}\PY{p}{,} \PY{n}{square}\PY{o}{=}\PY{k+kc}{True}\PY{p}{,} \PY{n}{cbar}\PY{o}{=}\PY{k+kc}{True}\PY{p}{,} \PY{n}{annot} \PY{o}{=} \PY{k+kc}{True}\PY{p}{,} \PY{n}{cmap}\PY{o}{=}\PY{l+s+s1}{\PYZsq{}}\PY{l+s+s1}{Spectral}\PY{l+s+s1}{\PYZsq{}}\PY{p}{)}
          \PY{n}{plt}\PY{o}{.}\PY{n}{show}\PY{p}{(}\PY{p}{)}
\end{Verbatim}

    \begin{center}
    \adjustimage{max size={0.9\linewidth}{0.9\paperheight}}{output_82_0.png}
    \end{center}
    { \hspace*{\fill} \\}
    
    \hypertarget{atributos-categuxf3ricos}{%
\subsubsection{Atributos Categóricos}\label{atributos-categuxf3ricos}}

    Para tratar os atributos categóricos, observou-se, inicialmente, a
relação de cada uma das categorias com o preço e o tamanho da variância
desta estimativa média.

    \begin{Verbatim}[commandchars=\\\{\}]
{\color{incolor}In [{\color{incolor}233}]:} \PY{c+c1}{\PYZsh{} Relacionando os atributos literais ao preco, com visualizacao}
          
          \PY{c+c1}{\PYZsh{} Analisando a influencia de cut}
          \PY{n}{sns}\PY{o}{.}\PY{n}{barplot}\PY{p}{(}\PY{n}{x} \PY{o}{=} \PY{l+s+s2}{\PYZdq{}}\PY{l+s+s2}{price}\PY{l+s+s2}{\PYZdq{}}\PY{p}{,} \PY{n}{y} \PY{o}{=} \PY{l+s+s2}{\PYZdq{}}\PY{l+s+s2}{cut}\PY{l+s+s2}{\PYZdq{}}\PY{p}{,} \PY{n}{data} \PY{o}{=} \PY{n}{train}\PY{p}{)}
          \PY{n}{plt}\PY{o}{.}\PY{n}{show}\PY{p}{(}\PY{p}{)}
          
          \PY{c+c1}{\PYZsh{} Analisando a influencia de color}
          \PY{n}{sns}\PY{o}{.}\PY{n}{barplot}\PY{p}{(}\PY{n}{x} \PY{o}{=} \PY{l+s+s2}{\PYZdq{}}\PY{l+s+s2}{price}\PY{l+s+s2}{\PYZdq{}}\PY{p}{,} \PY{n}{y} \PY{o}{=} \PY{l+s+s2}{\PYZdq{}}\PY{l+s+s2}{color}\PY{l+s+s2}{\PYZdq{}}\PY{p}{,} \PY{n}{data} \PY{o}{=} \PY{n}{train}\PY{p}{)}
          \PY{n}{plt}\PY{o}{.}\PY{n}{show}\PY{p}{(}\PY{p}{)}
          
          \PY{c+c1}{\PYZsh{} Analisando a influencia de clarity}
          \PY{n}{sns}\PY{o}{.}\PY{n}{barplot}\PY{p}{(}\PY{n}{x} \PY{o}{=} \PY{l+s+s2}{\PYZdq{}}\PY{l+s+s2}{price}\PY{l+s+s2}{\PYZdq{}}\PY{p}{,} \PY{n}{y} \PY{o}{=} \PY{l+s+s2}{\PYZdq{}}\PY{l+s+s2}{clarity}\PY{l+s+s2}{\PYZdq{}}\PY{p}{,} \PY{n}{data} \PY{o}{=} \PY{n}{train}\PY{p}{)}
          \PY{n}{plt}\PY{o}{.}\PY{n}{show}\PY{p}{(}\PY{p}{)}
\end{Verbatim}

    \begin{center}
    \adjustimage{max size={0.9\linewidth}{0.9\paperheight}}{output_85_0.png}
    \end{center}
    { \hspace*{\fill} \\}
    
    \begin{center}
    \adjustimage{max size={0.9\linewidth}{0.9\paperheight}}{output_85_1.png}
    \end{center}
    { \hspace*{\fill} \\}
    
    \begin{center}
    \adjustimage{max size={0.9\linewidth}{0.9\paperheight}}{output_85_2.png}
    \end{center}
    { \hspace*{\fill} \\}
    
    Como pode ser observado nos gráficos acima, as categorias de cada um dos
atributos influenciam o preço de maneiras diferentes. As barras
coloridas significam o valor estimado para cada opção e a linha ao final
de cada barra informa a incerteza destas estimativas.

    Por conta deste fenômeno, decidiu-se lidar com estes atributos da
seguinte maneira: 1. Utilizando \emph{Hot Enconding}, ou seja, dividindo
cada um deles em novos atributos booleanos referentes às suas categorias
através da função \emph{get\_dummies()}. Esta alternativa foi descartada
e não está presente neste documento, pois todos os seus resultados
apresentaram comportamento piorado em relação aos demais. 2.
Substituindo cada uma das categorias por um valor inteiro que
representasse a contribuição em relação ao preço. Ou seja, categorias
com maior valor estimado de preço deveriam possuir maior valor inteiro
de substituição e categorias com menor valor estimado, menor. Este
procedimento foi realizado de duas maneiras diferentes: \textgreater{}
1. Considerando valores de 1 até o número de categorias de cada atributo
categórico; \textgreater{} 2. Considerando as médias de preço por
categoria.

As duas últimas opções foram escolhidas pois apresentaram melhores
resultados e podem ser observadas abaixo:

    \begin{Verbatim}[commandchars=\\\{\}]
{\color{incolor}In [{\color{incolor}234}]:} \PY{c+c1}{\PYZsh{} Numeros inteiros de 1 a numero de categorias do atributo}
          \PY{c+c1}{\PYZsh{} Valores escolhidos de acordo com observacao dos graficos}
          
          \PY{c+c1}{\PYZsh{} cut}
          \PY{n}{train}\PY{p}{[}\PY{l+s+s1}{\PYZsq{}}\PY{l+s+s1}{cut}\PY{l+s+s1}{\PYZsq{}}\PY{p}{]} \PY{o}{=} \PY{n}{train}\PY{p}{[}\PY{l+s+s1}{\PYZsq{}}\PY{l+s+s1}{cut}\PY{l+s+s1}{\PYZsq{}}\PY{p}{]}\PY{o}{.}\PY{n}{replace}\PY{p}{(}\PY{p}{\PYZob{}}\PY{l+s+s1}{\PYZsq{}}\PY{l+s+s1}{Ideal}\PY{l+s+s1}{\PYZsq{}}\PY{p}{:} \PY{l+m+mi}{1}\PY{p}{,} \PY{l+s+s1}{\PYZsq{}}\PY{l+s+s1}{Good}\PY{l+s+s1}{\PYZsq{}}\PY{p}{:} \PY{l+m+mi}{2}\PY{p}{,} \PY{l+s+s1}{\PYZsq{}}\PY{l+s+s1}{Very Good}\PY{l+s+s1}{\PYZsq{}}\PY{p}{:} \PY{l+m+mi}{3}\PY{p}{,} \PY{l+s+s1}{\PYZsq{}}\PY{l+s+s1}{Fair}\PY{l+s+s1}{\PYZsq{}}\PY{p}{:} \PY{l+m+mi}{4}\PY{p}{,} \PY{l+s+s1}{\PYZsq{}}\PY{l+s+s1}{Premium}\PY{l+s+s1}{\PYZsq{}}\PY{p}{:} \PY{l+m+mi}{5}\PY{p}{\PYZcb{}}\PY{p}{)}
          \PY{n}{test}\PY{p}{[}\PY{l+s+s1}{\PYZsq{}}\PY{l+s+s1}{cut}\PY{l+s+s1}{\PYZsq{}}\PY{p}{]}  \PY{o}{=} \PY{n}{test}\PY{p}{[}\PY{l+s+s1}{\PYZsq{}}\PY{l+s+s1}{cut}\PY{l+s+s1}{\PYZsq{}}\PY{p}{]}\PY{o}{.}\PY{n}{replace}\PY{p}{(}\PY{p}{\PYZob{}}\PY{l+s+s1}{\PYZsq{}}\PY{l+s+s1}{Ideal}\PY{l+s+s1}{\PYZsq{}}\PY{p}{:} \PY{l+m+mi}{1}\PY{p}{,} \PY{l+s+s1}{\PYZsq{}}\PY{l+s+s1}{Good}\PY{l+s+s1}{\PYZsq{}}\PY{p}{:} \PY{l+m+mi}{2}\PY{p}{,} \PY{l+s+s1}{\PYZsq{}}\PY{l+s+s1}{Very Good}\PY{l+s+s1}{\PYZsq{}}\PY{p}{:} \PY{l+m+mi}{3}\PY{p}{,} \PY{l+s+s1}{\PYZsq{}}\PY{l+s+s1}{Fair}\PY{l+s+s1}{\PYZsq{}}\PY{p}{:} \PY{l+m+mi}{4}\PY{p}{,} \PY{l+s+s1}{\PYZsq{}}\PY{l+s+s1}{Premium}\PY{l+s+s1}{\PYZsq{}}\PY{p}{:} \PY{l+m+mi}{5}\PY{p}{\PYZcb{}}\PY{p}{)}
          
          \PY{c+c1}{\PYZsh{} color}
          \PY{n}{train}\PY{p}{[}\PY{l+s+s1}{\PYZsq{}}\PY{l+s+s1}{color}\PY{l+s+s1}{\PYZsq{}}\PY{p}{]} \PY{o}{=} \PY{n}{train}\PY{p}{[}\PY{l+s+s1}{\PYZsq{}}\PY{l+s+s1}{color}\PY{l+s+s1}{\PYZsq{}}\PY{p}{]}\PY{o}{.}\PY{n}{replace}\PY{p}{(}\PY{p}{\PYZob{}}\PY{l+s+s1}{\PYZsq{}}\PY{l+s+s1}{E}\PY{l+s+s1}{\PYZsq{}}\PY{p}{:} \PY{l+m+mi}{1}\PY{p}{,} \PY{l+s+s1}{\PYZsq{}}\PY{l+s+s1}{D}\PY{l+s+s1}{\PYZsq{}}\PY{p}{:} \PY{l+m+mi}{2}\PY{p}{,} \PY{l+s+s1}{\PYZsq{}}\PY{l+s+s1}{F}\PY{l+s+s1}{\PYZsq{}}\PY{p}{:} \PY{l+m+mi}{3}\PY{p}{,} \PY{l+s+s1}{\PYZsq{}}\PY{l+s+s1}{G}\PY{l+s+s1}{\PYZsq{}}\PY{p}{:} \PY{l+m+mi}{4}\PY{p}{,} \PY{l+s+s1}{\PYZsq{}}\PY{l+s+s1}{H}\PY{l+s+s1}{\PYZsq{}}\PY{p}{:} \PY{l+m+mi}{5}\PY{p}{,} \PY{l+s+s1}{\PYZsq{}}\PY{l+s+s1}{I}\PY{l+s+s1}{\PYZsq{}}\PY{p}{:} \PY{l+m+mi}{6}\PY{p}{,} \PY{l+s+s1}{\PYZsq{}}\PY{l+s+s1}{J}\PY{l+s+s1}{\PYZsq{}}\PY{p}{:} \PY{l+m+mi}{7}\PY{p}{\PYZcb{}}\PY{p}{)}
          \PY{n}{test}\PY{p}{[}\PY{l+s+s1}{\PYZsq{}}\PY{l+s+s1}{color}\PY{l+s+s1}{\PYZsq{}}\PY{p}{]}  \PY{o}{=} \PY{n}{test}\PY{p}{[}\PY{l+s+s1}{\PYZsq{}}\PY{l+s+s1}{color}\PY{l+s+s1}{\PYZsq{}}\PY{p}{]}\PY{o}{.}\PY{n}{replace}\PY{p}{(}\PY{p}{\PYZob{}}\PY{l+s+s1}{\PYZsq{}}\PY{l+s+s1}{E}\PY{l+s+s1}{\PYZsq{}}\PY{p}{:} \PY{l+m+mi}{1}\PY{p}{,} \PY{l+s+s1}{\PYZsq{}}\PY{l+s+s1}{D}\PY{l+s+s1}{\PYZsq{}}\PY{p}{:} \PY{l+m+mi}{2}\PY{p}{,} \PY{l+s+s1}{\PYZsq{}}\PY{l+s+s1}{F}\PY{l+s+s1}{\PYZsq{}}\PY{p}{:} \PY{l+m+mi}{3}\PY{p}{,} \PY{l+s+s1}{\PYZsq{}}\PY{l+s+s1}{G}\PY{l+s+s1}{\PYZsq{}}\PY{p}{:} \PY{l+m+mi}{4}\PY{p}{,} \PY{l+s+s1}{\PYZsq{}}\PY{l+s+s1}{H}\PY{l+s+s1}{\PYZsq{}}\PY{p}{:} \PY{l+m+mi}{5}\PY{p}{,} \PY{l+s+s1}{\PYZsq{}}\PY{l+s+s1}{I}\PY{l+s+s1}{\PYZsq{}}\PY{p}{:} \PY{l+m+mi}{6}\PY{p}{,} \PY{l+s+s1}{\PYZsq{}}\PY{l+s+s1}{J}\PY{l+s+s1}{\PYZsq{}}\PY{p}{:} \PY{l+m+mi}{7}\PY{p}{\PYZcb{}}\PY{p}{)}
          
          \PY{c+c1}{\PYZsh{} clarity}
          \PY{n}{train}\PY{p}{[}\PY{l+s+s1}{\PYZsq{}}\PY{l+s+s1}{clarity}\PY{l+s+s1}{\PYZsq{}}\PY{p}{]} \PY{o}{=} \PY{n}{train}\PY{p}{[}\PY{l+s+s1}{\PYZsq{}}\PY{l+s+s1}{clarity}\PY{l+s+s1}{\PYZsq{}}\PY{p}{]}\PY{o}{.}\PY{n}{replace}\PY{p}{(}\PY{p}{\PYZob{}}\PY{l+s+s1}{\PYZsq{}}\PY{l+s+s1}{VVS1}\PY{l+s+s1}{\PYZsq{}}\PY{p}{:} \PY{l+m+mi}{1}\PY{p}{,} \PY{l+s+s1}{\PYZsq{}}\PY{l+s+s1}{IF}\PY{l+s+s1}{\PYZsq{}}\PY{p}{:} \PY{l+m+mi}{2}\PY{p}{,} \PY{l+s+s1}{\PYZsq{}}\PY{l+s+s1}{VVS2}\PY{l+s+s1}{\PYZsq{}}\PY{p}{:} \PY{l+m+mi}{3}\PY{p}{,} \PY{l+s+s1}{\PYZsq{}}\PY{l+s+s1}{VS1}\PY{l+s+s1}{\PYZsq{}}\PY{p}{:} \PY{l+m+mi}{4}\PY{p}{,} \PY{l+s+s1}{\PYZsq{}}\PY{l+s+s1}{VS2}\PY{l+s+s1}{\PYZsq{}}\PY{p}{:} \PY{l+m+mi}{5}\PY{p}{,} \PYZbs{}
                                                       \PY{l+s+s1}{\PYZsq{}}\PY{l+s+s1}{SI1}\PY{l+s+s1}{\PYZsq{}}\PY{p}{:} \PY{l+m+mi}{6}\PY{p}{,} \PY{l+s+s1}{\PYZsq{}}\PY{l+s+s1}{I1}\PY{l+s+s1}{\PYZsq{}}\PY{p}{:} \PY{l+m+mi}{7}\PY{p}{,} \PY{l+s+s1}{\PYZsq{}}\PY{l+s+s1}{SI2}\PY{l+s+s1}{\PYZsq{}}\PY{p}{:} \PY{l+m+mi}{8}\PY{p}{\PYZcb{}}\PY{p}{)}
          \PY{n}{test}\PY{p}{[}\PY{l+s+s1}{\PYZsq{}}\PY{l+s+s1}{clarity}\PY{l+s+s1}{\PYZsq{}}\PY{p}{]}  \PY{o}{=} \PY{n}{test}\PY{p}{[}\PY{l+s+s1}{\PYZsq{}}\PY{l+s+s1}{clarity}\PY{l+s+s1}{\PYZsq{}}\PY{p}{]}\PY{o}{.}\PY{n}{replace}\PY{p}{(}\PY{p}{\PYZob{}}\PY{l+s+s1}{\PYZsq{}}\PY{l+s+s1}{VVS1}\PY{l+s+s1}{\PYZsq{}}\PY{p}{:} \PY{l+m+mi}{1}\PY{p}{,} \PY{l+s+s1}{\PYZsq{}}\PY{l+s+s1}{IF}\PY{l+s+s1}{\PYZsq{}}\PY{p}{:} \PY{l+m+mi}{2}\PY{p}{,} \PY{l+s+s1}{\PYZsq{}}\PY{l+s+s1}{VVS2}\PY{l+s+s1}{\PYZsq{}}\PY{p}{:} \PY{l+m+mi}{3}\PY{p}{,} \PY{l+s+s1}{\PYZsq{}}\PY{l+s+s1}{VS1}\PY{l+s+s1}{\PYZsq{}}\PY{p}{:} \PY{l+m+mi}{4}\PY{p}{,} \PY{l+s+s1}{\PYZsq{}}\PY{l+s+s1}{VS2}\PY{l+s+s1}{\PYZsq{}}\PY{p}{:} \PY{l+m+mi}{5}\PY{p}{,} \PYZbs{}
                                                      \PY{l+s+s1}{\PYZsq{}}\PY{l+s+s1}{SI1}\PY{l+s+s1}{\PYZsq{}}\PY{p}{:} \PY{l+m+mi}{6}\PY{p}{,} \PY{l+s+s1}{\PYZsq{}}\PY{l+s+s1}{I1}\PY{l+s+s1}{\PYZsq{}}\PY{p}{:} \PY{l+m+mi}{7}\PY{p}{,} \PY{l+s+s1}{\PYZsq{}}\PY{l+s+s1}{SI2}\PY{l+s+s1}{\PYZsq{}}\PY{p}{:} \PY{l+m+mi}{8}\PY{p}{\PYZcb{}}\PY{p}{)}
          
          \PY{c+c1}{\PYZsh{} Visualizando teste para checar funcionamento}
          \PY{n}{test}\PY{o}{.}\PY{n}{head}\PY{p}{(}\PY{p}{)}
\end{Verbatim}

\begin{Verbatim}[commandchars=\\\{\}]
{\color{outcolor}Out[{\color{outcolor}234}]:}     carat  cut  color  clarity     x     y     z  depth  table
          id                                                            
          0    1.82    5      4        6  7.75  7.68  4.84   62.7   58.0
          1    1.11    3      5        6  6.63  6.65  4.11   61.9   58.0
          2    0.52    1      2        3  5.19  5.22  3.20   61.5   55.0
          3    1.05    1      6        5  6.52  6.50  4.05   62.2   56.0
          4    0.70    1      6        3  5.63  5.68  3.51   62.1   58.0
\end{Verbatim}
            
    \begin{Verbatim}[commandchars=\\\{\}]
{\color{incolor}In [{\color{incolor}235}]:} \PY{c+c1}{\PYZsh{} \PYZsh{} Funcao de transformacao dos atributos categoricos}
          \PY{c+c1}{\PYZsh{} \PYZsh{} nas medias de preco por categoria}
          
          \PY{c+c1}{\PYZsh{} def categ\PYZus{}feature(feature, data):}
          \PY{c+c1}{\PYZsh{}     mean = train.groupby(feature)[\PYZsq{}price\PYZsq{}].mean()}
          \PY{c+c1}{\PYZsh{}     mean\PYZus{}sort = mean.reset\PYZus{}index().sort\PYZus{}values([\PYZsq{}price\PYZsq{}]).set\PYZus{}index([feature]).astype(int)}
              
          \PY{c+c1}{\PYZsh{}     mean\PYZus{}sort.to\PYZus{}dict()}
          \PY{c+c1}{\PYZsh{}     mean\PYZus{}sort = mean\PYZus{}sort[\PYZsq{}price\PYZsq{}]}
              
          \PY{c+c1}{\PYZsh{}     data[feature] = data[feature].replace(mean\PYZus{}sort, inplace = False)}
              
          \PY{c+c1}{\PYZsh{}     return mean\PYZus{}sort, data}
\end{Verbatim}

    \begin{Verbatim}[commandchars=\\\{\}]
{\color{incolor}In [{\color{incolor}236}]:} \PY{c+c1}{\PYZsh{} \PYZsh{} Aplicando a funcao para os dados de treino e teste}
          
          \PY{c+c1}{\PYZsh{} \PYZsh{} cut}
          \PY{c+c1}{\PYZsh{} mean\PYZus{}sort\PYZus{}cut, train = categ\PYZus{}feature(\PYZsq{}cut\PYZsq{}, train)}
          \PY{c+c1}{\PYZsh{} test[\PYZsq{}cut\PYZsq{}] = test[\PYZsq{}cut\PYZsq{}].replace(mean\PYZus{}sort\PYZus{}cut, inplace = False)}
          
          \PY{c+c1}{\PYZsh{} \PYZsh{}color}
          \PY{c+c1}{\PYZsh{} mean\PYZus{}sort\PYZus{}color, train = categ\PYZus{}feature(\PYZsq{}color\PYZsq{}, train)}
          \PY{c+c1}{\PYZsh{} test[\PYZsq{}color\PYZsq{}] = test[\PYZsq{}color\PYZsq{}].replace(mean\PYZus{}sort\PYZus{}color, inplace = False)}
          
          \PY{c+c1}{\PYZsh{} \PYZsh{}clarity}
          \PY{c+c1}{\PYZsh{} mean\PYZus{}sort\PYZus{}clarity, train = categ\PYZus{}feature(\PYZsq{}clarity\PYZsq{}, train)}
          \PY{c+c1}{\PYZsh{} test[\PYZsq{}clarity\PYZsq{}] = test[\PYZsq{}clarity\PYZsq{}].replace(mean\PYZus{}sort\PYZus{}clarity, inplace = False)}
          
          \PY{c+c1}{\PYZsh{} \PYZsh{} Visualizando teste para checar funcionamento}
          \PY{c+c1}{\PYZsh{} test.head()}
\end{Verbatim}

    \begin{Verbatim}[commandchars=\\\{\}]
{\color{incolor}In [{\color{incolor}237}]:} \PY{c+c1}{\PYZsh{} Criando a matriz de correlacao novamente para analise}
          \PY{n}{corr\PYZus{}matrix} \PY{o}{=} \PY{n}{train}\PY{o}{.}\PY{n}{corr}\PY{p}{(}\PY{p}{)}
          
          \PY{n}{plt}\PY{o}{.}\PY{n}{subplots}\PY{p}{(}\PY{n}{figsize} \PY{o}{=} \PY{p}{(}\PY{l+m+mi}{10}\PY{p}{,} \PY{l+m+mi}{10}\PY{p}{)}\PY{p}{)}
          \PY{n}{sns}\PY{o}{.}\PY{n}{heatmap}\PY{p}{(}\PY{n}{corr\PYZus{}matrix}\PY{p}{,} \PY{n}{square}\PY{o}{=}\PY{k+kc}{True}\PY{p}{,} \PY{n}{cbar}\PY{o}{=}\PY{k+kc}{True}\PY{p}{,} \PY{n}{annot} \PY{o}{=} \PY{k+kc}{True}\PY{p}{,} \PY{n}{cmap}\PY{o}{=}\PY{l+s+s1}{\PYZsq{}}\PY{l+s+s1}{Spectral}\PY{l+s+s1}{\PYZsq{}}\PY{p}{)}
          \PY{n}{plt}\PY{o}{.}\PY{n}{show}\PY{p}{(}\PY{p}{)}
\end{Verbatim}

    \begin{center}
    \adjustimage{max size={0.9\linewidth}{0.9\paperheight}}{output_91_0.png}
    \end{center}
    { \hspace*{\fill} \\}
    
    A partir da matriz de correlação, é possível observar que os dados
transformados não possuem correlação alta com o preço. Entretanto,
dependendo do modelo a ser aplicado, eles podem apresentar influências
cruciais, provocando melhores métricas de ajuste fino do modelo, por
exemplo.

    \hypertarget{treino-e-anuxe1lise-de-resultados}{%
\section{Treino e Análise de
Resultados}\label{treino-e-anuxe1lise-de-resultados}}

    \hypertarget{preparauxe7uxe3o}{%
\subsection{Preparação}\label{preparauxe7uxe3o}}

    Após aplicar todos os tratamentos e entender como cada atributo interage
com o alvo, é necessário dividir o conjunto de treino em dois, um de
treino e um de pseudo-teste, a fim de avaliar o comportamento de cada
modelo para verificar qual é a melhor escolha para a aplicação em
questão.\\
É importante ressaltar que, para realizar esta separação, o conjunto de
treino deve ser previamente separado em duas partes específicas: x, com
todos os atributos, e y, com o alvo, como pode ser observado abaixo.

    \begin{Verbatim}[commandchars=\\\{\}]
{\color{incolor}In [{\color{incolor}238}]:} \PY{c+c1}{\PYZsh{} Criando o conjunto de treino e de teste para treinar o modelo a partir de train modificado}
          \PY{n}{x} \PY{o}{=} \PY{n}{train}\PY{o}{.}\PY{n}{drop}\PY{p}{(}\PY{p}{[}\PY{l+s+s1}{\PYZsq{}}\PY{l+s+s1}{price}\PY{l+s+s1}{\PYZsq{}}\PY{p}{]}\PY{p}{,} \PY{n}{axis} \PY{o}{=} \PY{l+m+mi}{1}\PY{p}{)}
          \PY{n}{y} \PY{o}{=} \PY{n}{train}\PY{p}{[}\PY{l+s+s1}{\PYZsq{}}\PY{l+s+s1}{price}\PY{l+s+s1}{\PYZsq{}}\PY{p}{]}
          
          \PY{c+c1}{\PYZsh{} Criacao de conjuntos de treino e pseudo\PYZhy{}teste a partir do conjunto geral de treino}
          \PY{n}{X\PYZus{}train}\PY{p}{,} \PY{n}{X\PYZus{}test}\PY{p}{,} \PY{n}{y\PYZus{}train}\PY{p}{,} \PY{n}{y\PYZus{}test} \PY{o}{=} \PY{n}{train\PYZus{}test\PYZus{}split}\PY{p}{(}\PY{n}{x}\PY{p}{,} \PY{n}{y}\PY{p}{,} \PY{n}{random\PYZus{}state} \PY{o}{=} \PY{l+m+mi}{2}\PY{p}{,} \PY{n}{test\PYZus{}size}\PY{o}{=}\PY{l+m+mf}{0.3}\PY{p}{)}
\end{Verbatim}

    Cria-se, adicionalmente, um dicionário de comparação que irá armazenar
todos os RMSPE analisados durante os próximos passos. Este dicionário
servirá para demonstrar o modelo que gerou o melhor resultado.

Além dele, criam-se também duas funções, uma responsável pelo cálculo do
RMSPE, já que esta métrica não estava disponível nos modelos utilizados,
e uma responsável pela análise de cada modelo, capaz de gerar o
treinamento, predição e calcular as métricas de cada caso.

    \begin{Verbatim}[commandchars=\\\{\}]
{\color{incolor}In [{\color{incolor}239}]:} \PY{c+c1}{\PYZsh{} Arrays de referência para comparação entre modelos}
          \PY{n}{model\PYZus{}dict} \PY{o}{=} \PY{p}{\PYZob{}}\PY{l+s+s1}{\PYZsq{}}\PY{l+s+s1}{Linear Regressor}\PY{l+s+s1}{\PYZsq{}}\PY{p}{:} \PY{l+m+mi}{1}\PY{p}{,} \PY{l+s+s1}{\PYZsq{}}\PY{l+s+s1}{Lasso Regression}\PY{l+s+s1}{\PYZsq{}}\PY{p}{:} \PY{l+m+mi}{1}\PY{p}{,} \PY{l+s+s1}{\PYZsq{}}\PY{l+s+s1}{Ridge Regression}\PY{l+s+s1}{\PYZsq{}}\PY{p}{:} \PY{l+m+mi}{1}\PY{p}{,} \PY{l+s+s1}{\PYZsq{}}\PY{l+s+s1}{AdaBoost Regression}\PY{l+s+s1}{\PYZsq{}}\PY{p}{:} \PY{l+m+mi}{1}\PY{p}{,} \PYZbs{}
                      \PY{l+s+s1}{\PYZsq{}}\PY{l+s+s1}{Gradient Boosting Regression}\PY{l+s+s1}{\PYZsq{}}\PY{p}{:} \PY{l+m+mi}{1}\PY{p}{,} \PY{l+s+s1}{\PYZsq{}}\PY{l+s+s1}{Random Forest Regression}\PY{l+s+s1}{\PYZsq{}}\PY{p}{:} \PY{l+m+mi}{1}\PY{p}{,} \PY{l+s+s1}{\PYZsq{}}\PY{l+s+s1}{Extra Trees Regression}\PY{l+s+s1}{\PYZsq{}}\PY{p}{:} \PY{l+m+mi}{1}\PY{p}{\PYZcb{}}
\end{Verbatim}

    \begin{Verbatim}[commandchars=\\\{\}]
{\color{incolor}In [{\color{incolor}240}]:} \PY{c+c1}{\PYZsh{} Função que calcula o RMSPE para validacao dos modelos}
          \PY{k}{def} \PY{n+nf}{rmspe\PYZus{}score}\PY{p}{(}\PY{n}{y\PYZus{}test}\PY{p}{,} \PY{n}{y\PYZus{}pred}\PY{p}{)}\PY{p}{:}
              
              \PY{n}{rmspe} \PY{o}{=} \PY{n}{np}\PY{o}{.}\PY{n}{sqrt}\PY{p}{(}\PY{n}{np}\PY{o}{.}\PY{n}{mean}\PY{p}{(}\PY{n}{np}\PY{o}{.}\PY{n}{square}\PY{p}{(}\PY{p}{(}\PY{p}{(}\PY{n}{y\PYZus{}test} \PY{o}{\PYZhy{}} \PY{n}{y\PYZus{}pred}\PY{p}{)} \PY{o}{/} \PY{n}{y\PYZus{}test}\PY{p}{)}\PY{p}{)}\PY{p}{,} \PY{n}{axis} \PY{o}{=} \PY{l+m+mi}{0}\PY{p}{)}\PY{p}{)}
          
              \PY{k}{return} \PY{n}{rmspe}
\end{Verbatim}

    \begin{Verbatim}[commandchars=\\\{\}]
{\color{incolor}In [{\color{incolor}241}]:} \PY{c+c1}{\PYZsh{} Funcao de regressao generica, para varios modelos diferentes}
          \PY{k}{def} \PY{n+nf}{model\PYZus{}analysis}\PY{p}{(}\PY{n}{X\PYZus{}train}\PY{p}{,} \PY{n}{X\PYZus{}test}\PY{p}{,} \PY{n}{y\PYZus{}train}\PY{p}{,} \PY{n}{y\PYZus{}test}\PY{p}{,} \PY{n}{regressor}\PY{p}{,} \PY{n}{name}\PY{p}{)}\PY{p}{:}
              \PY{n}{regressor}\PY{o}{.}\PY{n}{fit}\PY{p}{(}\PY{n}{X\PYZus{}train}\PY{p}{,} \PY{n}{y\PYZus{}train}\PY{p}{)}
              \PY{n}{y\PYZus{}pred} \PY{o}{=} \PY{n}{regressor}\PY{o}{.}\PY{n}{predict}\PY{p}{(}\PY{n}{X\PYZus{}test}\PY{p}{)}
              \PY{n+nb}{print}\PY{p}{(}\PY{l+s+s1}{\PYZsq{}}\PY{l+s+s1}{\PYZsq{}}\PY{p}{)}
              \PY{n+nb}{print}\PY{p}{(}\PY{l+s+s1}{\PYZsq{}}\PY{l+s+s1}{\PYZsh{}\PYZsh{}\PYZsh{}\PYZsh{}\PYZsh{}\PYZsh{} }\PY{l+s+si}{\PYZob{}\PYZcb{}}\PY{l+s+s1}{ \PYZsh{}\PYZsh{}\PYZsh{}\PYZsh{}\PYZsh{}\PYZsh{}}\PY{l+s+s1}{\PYZsq{}}\PY{o}{.}\PY{n}{format}\PY{p}{(}\PY{n}{name}\PY{p}{)}\PY{p}{)}
              \PY{n+nb}{print}\PY{p}{(}\PY{l+s+s1}{\PYZsq{}}\PY{l+s+s1}{Score : }\PY{l+s+si}{\PYZpc{}.6f}\PY{l+s+s1}{\PYZsq{}} \PY{o}{\PYZpc{}} \PY{n}{regressor}\PY{o}{.}\PY{n}{score}\PY{p}{(}\PY{n}{X\PYZus{}test}\PY{p}{,} \PY{n}{y\PYZus{}test}\PY{p}{)}\PY{p}{)}
          
              \PY{n}{mse} \PY{o}{=} \PY{n}{mean\PYZus{}squared\PYZus{}error}\PY{p}{(}\PY{n}{y\PYZus{}test}\PY{p}{,} \PY{n}{y\PYZus{}pred}\PY{p}{)}
              \PY{n}{mae} \PY{o}{=} \PY{n}{mean\PYZus{}absolute\PYZus{}error}\PY{p}{(}\PY{n}{y\PYZus{}test}\PY{p}{,} \PY{n}{y\PYZus{}pred}\PY{p}{)}
              \PY{n}{rmse} \PY{o}{=} \PY{n}{mean\PYZus{}squared\PYZus{}error}\PY{p}{(}\PY{n}{y\PYZus{}test}\PY{p}{,} \PY{n}{y\PYZus{}pred}\PY{p}{)} \PY{o}{*}\PY{o}{*} \PY{l+m+mf}{0.5}
              \PY{n}{r2} \PY{o}{=} \PY{n}{r2\PYZus{}score}\PY{p}{(}\PY{n}{y\PYZus{}test}\PY{p}{,} \PY{n}{y\PYZus{}pred}\PY{p}{)}
              \PY{n}{rmspe} \PY{o}{=} \PY{n}{rmspe\PYZus{}score}\PY{p}{(}\PY{n}{y\PYZus{}test}\PY{p}{,} \PY{n}{y\PYZus{}pred}\PY{p}{)}
          
              \PY{n+nb}{print}\PY{p}{(}\PY{l+s+s1}{\PYZsq{}}\PY{l+s+s1}{\PYZsq{}}\PY{p}{)}
              \PY{n+nb}{print}\PY{p}{(}\PY{l+s+s1}{\PYZsq{}}\PY{l+s+s1}{MSE   : }\PY{l+s+si}{\PYZpc{}0.6f}\PY{l+s+s1}{ }\PY{l+s+s1}{\PYZsq{}} \PY{o}{\PYZpc{}} \PY{n}{mse}\PY{p}{)}
              \PY{n+nb}{print}\PY{p}{(}\PY{l+s+s1}{\PYZsq{}}\PY{l+s+s1}{MAE   : }\PY{l+s+si}{\PYZpc{}0.6f}\PY{l+s+s1}{ }\PY{l+s+s1}{\PYZsq{}} \PY{o}{\PYZpc{}} \PY{n}{mae}\PY{p}{)}
              \PY{n+nb}{print}\PY{p}{(}\PY{l+s+s1}{\PYZsq{}}\PY{l+s+s1}{RMSE  : }\PY{l+s+si}{\PYZpc{}0.6f}\PY{l+s+s1}{ }\PY{l+s+s1}{\PYZsq{}} \PY{o}{\PYZpc{}} \PY{n}{rmse}\PY{p}{)}
              \PY{n+nb}{print}\PY{p}{(}\PY{l+s+s1}{\PYZsq{}}\PY{l+s+s1}{R2    : }\PY{l+s+si}{\PYZpc{}0.6f}\PY{l+s+s1}{ }\PY{l+s+s1}{\PYZsq{}} \PY{o}{\PYZpc{}} \PY{n}{r2}\PY{p}{)}
              \PY{n+nb}{print}\PY{p}{(}\PY{l+s+s1}{\PYZsq{}}\PY{l+s+s1}{RMSPE : }\PY{l+s+si}{\PYZpc{}0.6f}\PY{l+s+s1}{ }\PY{l+s+s1}{\PYZsq{}} \PY{o}{\PYZpc{}} \PY{n}{rmspe}\PY{p}{)}
              
              \PY{n}{model\PYZus{}dict}\PY{p}{[}\PY{n}{name}\PY{p}{]} \PY{o}{=} \PY{n+nb}{round}\PY{p}{(}\PY{n}{rmspe}\PY{p}{,} \PY{l+m+mi}{6}\PY{p}{)}
\end{Verbatim}

    \hypertarget{modelos-testados}{%
\subsection{Modelos Testados}\label{modelos-testados}}

    A partir da preparação do ambiente, aplicam-se os mesmos dados para
todos os regressores abaixo, comparando os scores e todas as métricas de
erros. A métrica mais importante para este caso é o RMSPE, já que o
mesmo é o que será considerado como avaliador durante a competição.

    \hypertarget{linear-regression}{%
\subsubsection{Linear Regression}\label{linear-regression}}

    \begin{Verbatim}[commandchars=\\\{\}]
{\color{incolor}In [{\color{incolor}242}]:} \PY{o}{\PYZpc{}\PYZpc{}time}
          
          \PY{n}{lr} \PY{o}{=} \PY{n}{LinearRegression}\PY{p}{(}\PY{p}{)}
          \PY{n}{model\PYZus{}analysis}\PY{p}{(}\PY{n}{X\PYZus{}train}\PY{p}{,} \PY{n}{X\PYZus{}test}\PY{p}{,} \PY{n}{y\PYZus{}train}\PY{p}{,} \PY{n}{y\PYZus{}test}\PY{p}{,} \PY{n}{lr}\PY{p}{,} \PY{l+s+s1}{\PYZsq{}}\PY{l+s+s1}{Linear Regressor}\PY{l+s+s1}{\PYZsq{}}\PY{p}{)}
\end{Verbatim}

    \begin{Verbatim}[commandchars=\\\{\}]

\#\#\#\#\#\# Linear Regressor \#\#\#\#\#\#
Score : 0.901517

MSE   : 1554387.890560 
MAE   : 806.085542 
RMSE  : 1246.750934 
R2    : 0.901517 
RMSPE : 0.739535 
Wall time: 539 ms

    \end{Verbatim}

    \hypertarget{lasso-regression}{%
\subsubsection{Lasso Regression}\label{lasso-regression}}

    \begin{Verbatim}[commandchars=\\\{\}]
{\color{incolor}In [{\color{incolor}243}]:} \PY{o}{\PYZpc{}\PYZpc{}time}
          
          \PY{n}{lar} \PY{o}{=} \PY{n}{Lasso}\PY{p}{(}\PY{n}{normalize} \PY{o}{=} \PY{k+kc}{True}\PY{p}{)}
          \PY{n}{model\PYZus{}analysis}\PY{p}{(}\PY{n}{X\PYZus{}train}\PY{p}{,} \PY{n}{X\PYZus{}test}\PY{p}{,} \PY{n}{y\PYZus{}train}\PY{p}{,} \PY{n}{y\PYZus{}test}\PY{p}{,} \PY{n}{lar}\PY{p}{,} \PY{l+s+s1}{\PYZsq{}}\PY{l+s+s1}{Lasso Regression}\PY{l+s+s1}{\PYZsq{}}\PY{p}{)}
\end{Verbatim}

    \begin{Verbatim}[commandchars=\\\{\}]

\#\#\#\#\#\# Lasso Regression \#\#\#\#\#\#
Score : 0.890139

MSE   : 1733969.560359 
MAE   : 836.359696 
RMSE  : 1316.802780 
R2    : 0.890139 
RMSPE : 0.553080 
Wall time: 45.9 ms

    \end{Verbatim}

    \hypertarget{ridge-regression}{%
\subsubsection{Ridge Regression}\label{ridge-regression}}

    \begin{Verbatim}[commandchars=\\\{\}]
{\color{incolor}In [{\color{incolor}244}]:} \PY{o}{\PYZpc{}\PYZpc{}time}
          
          \PY{n}{rr} \PY{o}{=} \PY{n}{Ridge}\PY{p}{(}\PY{n}{normalize} \PY{o}{=} \PY{k+kc}{True}\PY{p}{)}
          \PY{n}{model\PYZus{}analysis}\PY{p}{(}\PY{n}{X\PYZus{}train}\PY{p}{,} \PY{n}{X\PYZus{}test}\PY{p}{,} \PY{n}{y\PYZus{}train}\PY{p}{,} \PY{n}{y\PYZus{}test}\PY{p}{,} \PY{n}{rr}\PY{p}{,} \PY{l+s+s1}{\PYZsq{}}\PY{l+s+s1}{Ridge Regression}\PY{l+s+s1}{\PYZsq{}}\PY{p}{)}
\end{Verbatim}

    \begin{Verbatim}[commandchars=\\\{\}]

\#\#\#\#\#\# Ridge Regression \#\#\#\#\#\#
Score : 0.799085

MSE   : 3171089.951095 
MAE   : 1142.106206 
RMSE  : 1780.755444 
R2    : 0.799085 
RMSPE : 0.483945 
Wall time: 144 ms

    \end{Verbatim}

    \hypertarget{adaboost-regression}{%
\subsubsection{AdaBoost Regression}\label{adaboost-regression}}

    \begin{Verbatim}[commandchars=\\\{\}]
{\color{incolor}In [{\color{incolor}245}]:} \PY{o}{\PYZpc{}\PYZpc{}time}
          
          \PY{n}{abr} \PY{o}{=} \PY{n}{AdaBoostRegressor}\PY{p}{(}\PY{n}{random\PYZus{}state} \PY{o}{=} \PY{l+m+mi}{2}\PY{p}{)}
          \PY{n}{model\PYZus{}analysis}\PY{p}{(}\PY{n}{X\PYZus{}train}\PY{p}{,} \PY{n}{X\PYZus{}test}\PY{p}{,} \PY{n}{y\PYZus{}train}\PY{p}{,} \PY{n}{y\PYZus{}test}\PY{p}{,} \PY{n}{abr}\PY{p}{,} \PY{l+s+s1}{\PYZsq{}}\PY{l+s+s1}{AdaBoost Regression}\PY{l+s+s1}{\PYZsq{}}\PY{p}{)}
\end{Verbatim}

    \begin{Verbatim}[commandchars=\\\{\}]

\#\#\#\#\#\# AdaBoost Regression \#\#\#\#\#\#
Score : 0.919645

MSE   : 1268270.423040 
MAE   : 871.986156 
RMSE  : 1126.175130 
R2    : 0.919645 
RMSPE : 0.669976 
Wall time: 1.1 s

    \end{Verbatim}

    \hypertarget{gradiente-boosting-regression}{%
\subsubsection{Gradiente Boosting
Regression}\label{gradiente-boosting-regression}}

    \begin{Verbatim}[commandchars=\\\{\}]
{\color{incolor}In [{\color{incolor}246}]:} \PY{o}{\PYZpc{}\PYZpc{}time}
          
          \PY{n}{gbr} \PY{o}{=} \PY{n}{GradientBoostingRegressor}\PY{p}{(}\PY{n}{n\PYZus{}estimators} \PY{o}{=} \PY{l+m+mi}{200}\PY{p}{,} \PY{n}{min\PYZus{}samples\PYZus{}leaf} \PY{o}{=} \PY{l+m+mi}{2}\PY{p}{,} \PY{n}{min\PYZus{}samples\PYZus{}split} \PY{o}{=} \PY{l+m+mi}{5}\PY{p}{,} \PYZbs{}
                                          \PY{n}{max\PYZus{}depth} \PY{o}{=} \PY{l+m+mi}{10}\PY{p}{,} \PY{n}{random\PYZus{}state} \PY{o}{=} \PY{l+m+mi}{2}\PY{p}{)}
          \PY{n}{model\PYZus{}analysis}\PY{p}{(}\PY{n}{X\PYZus{}train}\PY{p}{,} \PY{n}{X\PYZus{}test}\PY{p}{,} \PY{n}{y\PYZus{}train}\PY{p}{,} \PY{n}{y\PYZus{}test}\PY{p}{,} \PY{n}{gbr}\PY{p}{,} \PY{l+s+s1}{\PYZsq{}}\PY{l+s+s1}{Gradient Boosting Regression}\PY{l+s+s1}{\PYZsq{}}\PY{p}{)}
\end{Verbatim}

    \begin{Verbatim}[commandchars=\\\{\}]

\#\#\#\#\#\# Gradient Boosting Regression \#\#\#\#\#\#
Score : 0.979846

MSE   : 318092.619016 
MAE   : 277.866854 
RMSE  : 563.997003 
R2    : 0.979846 
RMSPE : 0.091436 
Wall time: 12.4 s

    \end{Verbatim}

    \hypertarget{random-forest-regression}{%
\subsubsection{Random Forest
Regression}\label{random-forest-regression}}

    \begin{Verbatim}[commandchars=\\\{\}]
{\color{incolor}In [{\color{incolor}247}]:} \PY{o}{\PYZpc{}\PYZpc{}time}
          
          \PY{n}{rfr} \PY{o}{=} \PY{n}{RandomForestRegressor}\PY{p}{(}\PY{n}{n\PYZus{}estimators} \PY{o}{=} \PY{l+m+mi}{250}\PY{p}{,} \PY{n}{n\PYZus{}jobs} \PY{o}{=} \PY{l+m+mi}{2}\PY{p}{,} \PY{n}{random\PYZus{}state} \PY{o}{=} \PY{l+m+mi}{2}\PY{p}{)}
          \PY{n}{model\PYZus{}analysis}\PY{p}{(}\PY{n}{X\PYZus{}train}\PY{p}{,} \PY{n}{X\PYZus{}test}\PY{p}{,} \PY{n}{y\PYZus{}train}\PY{p}{,} \PY{n}{y\PYZus{}test}\PY{p}{,} \PY{n}{rfr}\PY{p}{,} \PY{l+s+s1}{\PYZsq{}}\PY{l+s+s1}{Random Forest Regression}\PY{l+s+s1}{\PYZsq{}}\PY{p}{)}
\end{Verbatim}

    \begin{Verbatim}[commandchars=\\\{\}]

\#\#\#\#\#\# Random Forest Regression \#\#\#\#\#\#
Score : 0.980798

MSE   : 303072.960419 
MAE   : 278.160645 
RMSE  : 550.520627 
R2    : 0.980798 
RMSPE : 0.096114 
Wall time: 8.82 s

    \end{Verbatim}

    \hypertarget{extra-trees-regression}{%
\subsubsection{Extra Trees Regression}\label{extra-trees-regression}}

    \begin{Verbatim}[commandchars=\\\{\}]
{\color{incolor}In [{\color{incolor}248}]:} \PY{o}{\PYZpc{}\PYZpc{}time}
          
          \PY{n}{etr} \PY{o}{=} \PY{n}{ExtraTreesRegressor}\PY{p}{(}\PY{n}{n\PYZus{}estimators} \PY{o}{=} \PY{l+m+mi}{1000}\PY{p}{,} \PY{n}{n\PYZus{}jobs} \PY{o}{=} \PY{o}{\PYZhy{}}\PY{l+m+mi}{1}\PY{p}{,} \PY{n}{random\PYZus{}state} \PY{o}{=} \PY{l+m+mi}{2}\PY{p}{)}
          \PY{n}{model\PYZus{}analysis}\PY{p}{(}\PY{n}{X\PYZus{}train}\PY{p}{,} \PY{n}{X\PYZus{}test}\PY{p}{,} \PY{n}{y\PYZus{}train}\PY{p}{,} \PY{n}{y\PYZus{}test}\PY{p}{,} \PY{n}{etr}\PY{p}{,} \PY{l+s+s1}{\PYZsq{}}\PY{l+s+s1}{Extra Trees Regression}\PY{l+s+s1}{\PYZsq{}}\PY{p}{)}
\end{Verbatim}

    \begin{Verbatim}[commandchars=\\\{\}]

\#\#\#\#\#\# Extra Trees Regression \#\#\#\#\#\#
Score : 0.981144

MSE   : 297613.643041 
MAE   : 274.049600 
RMSE  : 545.539772 
R2    : 0.981144 
RMSPE : 0.098128 
Wall time: 13.4 s

    \end{Verbatim}

    \hypertarget{comparauxe7uxe3o}{%
\subsubsection{Comparação}\label{comparauxe7uxe3o}}

    \begin{Verbatim}[commandchars=\\\{\}]
{\color{incolor}In [{\color{incolor}249}]:} \PY{n}{compare} \PY{o}{=} \PY{n}{pd}\PY{o}{.}\PY{n}{DataFrame}\PY{p}{(}\PY{p}{)}
          \PY{n}{compare}\PY{p}{[}\PY{l+s+s1}{\PYZsq{}}\PY{l+s+s1}{Model}\PY{l+s+s1}{\PYZsq{}}\PY{p}{]} \PY{o}{=} \PY{n}{model\PYZus{}dict}\PY{o}{.}\PY{n}{keys}\PY{p}{(}\PY{p}{)}
          \PY{n}{compare}\PY{p}{[}\PY{l+s+s1}{\PYZsq{}}\PY{l+s+s1}{RMSPE}\PY{l+s+s1}{\PYZsq{}}\PY{p}{]} \PY{o}{=} \PY{n}{model\PYZus{}dict}\PY{o}{.}\PY{n}{values}\PY{p}{(}\PY{p}{)}
          
          \PY{n}{compare} \PY{o}{=} \PY{n}{compare}\PY{o}{.}\PY{n}{set\PYZus{}index}\PY{p}{(}\PY{l+s+s1}{\PYZsq{}}\PY{l+s+s1}{Model}\PY{l+s+s1}{\PYZsq{}}\PY{p}{)}\PY{o}{.}\PY{n}{sort\PYZus{}values}\PY{p}{(}\PY{p}{[}\PY{l+s+s1}{\PYZsq{}}\PY{l+s+s1}{RMSPE}\PY{l+s+s1}{\PYZsq{}}\PY{p}{]}\PY{p}{)}
          \PY{n}{compare}
\end{Verbatim}

\begin{Verbatim}[commandchars=\\\{\}]
{\color{outcolor}Out[{\color{outcolor}249}]:}                                  RMSPE
          Model                                 
          Gradient Boosting Regression  0.091436
          Random Forest Regression      0.096114
          Extra Trees Regression        0.098128
          Ridge Regression              0.483945
          Lasso Regression              0.553080
          AdaBoost Regression           0.669976
          Linear Regressor              0.739535
\end{Verbatim}
            
    Como o modelo que apresentou menor RMSPE foi o \emph{Gradient Boosting
Regressor}, o mesmo foi selecionado para sofrer otimizações de
parâmetros e ser retestado a cada nova descoberta, como pode ser
observado nos passos a seguir.

    \hypertarget{otimizauxe7uxe3o-de-paruxe2metros}{%
\subsection{Otimização de
Parâmetros}\label{otimizauxe7uxe3o-de-paruxe2metros}}

    A partir da escolha do modelo, pesquisas relacionadas e dados empíricos,
definiram-se parâmetros iniciais para aplicar dois métodos de otimização
e validação cruzada considerando 3 combinações diferentes dos conjuntos
de dados: 1. Random Search; 2. Grid Search.

    \hypertarget{random-search}{%
\subsubsection{Random Search}\label{random-search}}

    \begin{Verbatim}[commandchars=\\\{\}]
{\color{incolor}In [{\color{incolor}250}]:} \PY{c+c1}{\PYZsh{} Definindo a grid para aplicar RandomizedSearchCV}
          \PY{n}{n\PYZus{}estimators} \PY{o}{=} \PY{p}{[}\PY{n+nb}{int}\PY{p}{(}\PY{n}{x}\PY{p}{)} \PY{k}{for} \PY{n}{x} \PY{o+ow}{in} \PY{n}{np}\PY{o}{.}\PY{n}{linspace}\PY{p}{(}\PY{l+m+mi}{500}\PY{p}{,} \PY{l+m+mi}{2500}\PY{p}{,} \PY{l+m+mi}{5}\PY{p}{)}\PY{p}{]}
          \PY{n}{max\PYZus{}features} \PY{o}{=} \PY{p}{[}\PY{l+m+mi}{7}\PY{p}{,} \PY{l+m+mi}{8}\PY{p}{,} \PY{l+m+mi}{9}\PY{p}{]}
          \PY{n}{max\PYZus{}depth} \PY{o}{=} \PY{p}{[}\PY{l+m+mi}{10}\PY{p}{,} \PY{l+m+mi}{12}\PY{p}{,} \PY{l+m+mi}{14}\PY{p}{]}
          \PY{n}{min\PYZus{}samples\PYZus{}split} \PY{o}{=} \PY{p}{[}\PY{l+m+mi}{2}\PY{p}{,} \PY{l+m+mi}{3}\PY{p}{,} \PY{l+m+mi}{4}\PY{p}{]}
          \PY{n}{min\PYZus{}samples\PYZus{}leaf} \PY{o}{=} \PY{p}{[}\PY{l+m+mi}{2}\PY{p}{,} \PY{l+m+mi}{4}\PY{p}{,} \PY{l+m+mi}{6}\PY{p}{]}
          \PY{n}{bootstrap} \PY{o}{=} \PY{p}{[}\PY{k+kc}{True}\PY{p}{]}
          \PY{n}{random\PYZus{}state} \PY{o}{=} \PY{p}{[}\PY{l+m+mi}{2}\PY{p}{]}
          \PY{n}{learning\PYZus{}rate} \PY{o}{=} \PY{p}{[}\PY{n+nb}{round}\PY{p}{(}\PY{n+nb}{float}\PY{p}{(}\PY{n}{x}\PY{p}{)}\PY{p}{,} \PY{l+m+mi}{3}\PY{p}{)} \PY{k}{for} \PY{n}{x} \PY{o+ow}{in} \PY{n}{np}\PY{o}{.}\PY{n}{linspace}\PY{p}{(}\PY{l+m+mf}{0.01}\PY{p}{,} \PY{l+m+mf}{0.15}\PY{p}{,} \PY{l+m+mi}{15}\PY{p}{)}\PY{p}{]}
          
          \PY{n}{random\PYZus{}grid} \PY{o}{=} \PY{p}{\PYZob{}}\PY{l+s+s1}{\PYZsq{}}\PY{l+s+s1}{n\PYZus{}estimators}\PY{l+s+s1}{\PYZsq{}}\PY{p}{:} \PY{n}{n\PYZus{}estimators}\PY{p}{,}
                         \PY{l+s+s1}{\PYZsq{}}\PY{l+s+s1}{max\PYZus{}features}\PY{l+s+s1}{\PYZsq{}}\PY{p}{:} \PY{n}{max\PYZus{}features}\PY{p}{,}
                         \PY{l+s+s1}{\PYZsq{}}\PY{l+s+s1}{max\PYZus{}depth}\PY{l+s+s1}{\PYZsq{}}\PY{p}{:} \PY{n}{max\PYZus{}depth}\PY{p}{,}
                         \PY{l+s+s1}{\PYZsq{}}\PY{l+s+s1}{min\PYZus{}samples\PYZus{}split}\PY{l+s+s1}{\PYZsq{}}\PY{p}{:} \PY{n}{min\PYZus{}samples\PYZus{}split}\PY{p}{,}
                         \PY{l+s+s1}{\PYZsq{}}\PY{l+s+s1}{min\PYZus{}samples\PYZus{}leaf}\PY{l+s+s1}{\PYZsq{}}\PY{p}{:} \PY{n}{min\PYZus{}samples\PYZus{}leaf}\PY{p}{,}
                         \PY{l+s+s1}{\PYZsq{}}\PY{l+s+s1}{random\PYZus{}state}\PY{l+s+s1}{\PYZsq{}}\PY{p}{:} \PY{n}{random\PYZus{}state}\PY{p}{,}
                         \PY{l+s+s1}{\PYZsq{}}\PY{l+s+s1}{learning\PYZus{}rate}\PY{l+s+s1}{\PYZsq{}}\PY{p}{:} \PY{n}{learning\PYZus{}rate}\PY{p}{\PYZcb{}}
          
          \PY{n+nb}{print}\PY{p}{(}\PY{n}{random\PYZus{}grid}\PY{p}{)}
\end{Verbatim}

    \begin{Verbatim}[commandchars=\\\{\}]
\{'n\_estimators': [500, 1000, 1500, 2000, 2500], 'max\_features': [7, 8, 9], 'max\_depth': [10, 12, 14], 'min\_samples\_split': [2, 3, 4], 'min\_samples\_leaf': [2, 4, 6], 'random\_state': [2], 'learning\_rate': [0.01, 0.02, 0.03, 0.04, 0.05, 0.06, 0.07, 0.08, 0.09, 0.1, 0.11, 0.12, 0.13, 0.14, 0.15]\}

    \end{Verbatim}

    \begin{Verbatim}[commandchars=\\\{\}]
{\color{incolor}In [{\color{incolor}62}]:} \PY{c+c1}{\PYZsh{} Aplicando RandomizedSearchCV ao Gradient Boosting Regressor}
         
         \PY{c+c1}{\PYZsh{} Random search com 100 iterações}
         \PY{n}{gbr} \PY{o}{=} \PY{n}{GradientBoostingRegressor}\PY{p}{(}\PY{p}{)}
         \PY{n}{gbr\PYZus{}rs} \PY{o}{=} \PY{n}{RandomizedSearchCV}\PY{p}{(}\PY{n}{estimator} \PY{o}{=} \PY{n}{gbr}\PY{p}{,} 
                                     \PY{n}{param\PYZus{}distributions} \PY{o}{=} \PY{n}{random\PYZus{}grid}\PY{p}{,} 
                                     \PY{n}{n\PYZus{}iter} \PY{o}{=} \PY{l+m+mi}{100}\PY{p}{,} 
                                     \PY{n}{cv} \PY{o}{=} \PY{l+m+mi}{3}\PY{p}{,} 
                                     \PY{n}{verbose} \PY{o}{=} \PY{l+m+mi}{2}\PY{p}{,} 
                                     \PY{n}{random\PYZus{}state} \PY{o}{=} \PY{l+m+mi}{2}\PY{p}{,} 
                                     \PY{n}{n\PYZus{}jobs} \PY{o}{=} \PY{o}{\PYZhy{}}\PY{l+m+mi}{1}\PY{p}{,}
                                     \PY{n}{scoring} \PY{o}{=} \PY{n}{make\PYZus{}scorer}\PY{p}{(}\PY{n}{rmspe\PYZus{}score}\PY{p}{,} \PY{n}{greater\PYZus{}is\PYZus{}better} \PY{o}{=} \PY{k+kc}{False}\PY{p}{)}
         \PY{p}{)}
         
         \PY{c+c1}{\PYZsh{} Treina o modelo com 100 possibilidades aleatorias dentro do conjunto}
         \PY{c+c1}{\PYZsh{} definido pelo random\PYZus{}grid}
         \PY{n}{gbr\PYZus{}rs}\PY{o}{.}\PY{n}{fit}\PY{p}{(}\PY{n}{x}\PY{p}{,} \PY{n}{y}\PY{p}{)}
         
         \PY{c+c1}{\PYZsh{} Dentro destes treinamentos, define o que apresentou melhor resultado}
         \PY{n}{gbr\PYZus{}rs}\PY{o}{.}\PY{n}{best\PYZus{}params\PYZus{}}
\end{Verbatim}

    \begin{Verbatim}[commandchars=\\\{\}]
Fitting 3 folds for each of 100 candidates, totalling 300 fits

    \end{Verbatim}

    \begin{Verbatim}[commandchars=\\\{\}]
[Parallel(n\_jobs=-1)]: Using backend LokyBackend with 8 concurrent workers.
[Parallel(n\_jobs=-1)]: Done  25 tasks      | elapsed: 20.8min
[Parallel(n\_jobs=-1)]: Done 146 tasks      | elapsed: 113.7min
[Parallel(n\_jobs=-1)]: Done 300 out of 300 | elapsed: 261.1min finished

    \end{Verbatim}

\begin{Verbatim}[commandchars=\\\{\}]
{\color{outcolor}Out[{\color{outcolor}62}]:} \{'random\_state': 2,
          'n\_estimators': 500,
          'min\_samples\_split': 2,
          'min\_samples\_leaf': 6,
          'max\_features': 8,
          'max\_depth': 10,
          'learning\_rate': 0.04\}
\end{Verbatim}
            
    A partir dos parâmetros resultantes do método de Random Search, cria-se
uma nova grid em torno destes valores para gerar um novo conjunto de
teste. Este conjunto será responsável pela avaliação do método de Grid
Search.

    \hypertarget{grid-search}{%
\subsubsection{Grid Search}\label{grid-search}}

    \begin{Verbatim}[commandchars=\\\{\}]
{\color{incolor}In [{\color{incolor}56}]:} \PY{c+c1}{\PYZsh{} De acordo com os resultados da RandomizedSearch,}
         \PY{c+c1}{\PYZsh{} seto os parametros da param\PYZus{}grid em torno deles}
         \PY{c+c1}{\PYZsh{} para descobrir o melhor de todos}
         \PY{n}{x} \PY{o}{=} \PY{n}{train}\PY{o}{.}\PY{n}{drop}\PY{p}{(}\PY{p}{[}\PY{l+s+s1}{\PYZsq{}}\PY{l+s+s1}{price}\PY{l+s+s1}{\PYZsq{}}\PY{p}{]}\PY{p}{,} \PY{n}{axis} \PY{o}{=} \PY{l+m+mi}{1}\PY{p}{)}
         \PY{n}{y} \PY{o}{=} \PY{n}{train}\PY{p}{[}\PY{l+s+s1}{\PYZsq{}}\PY{l+s+s1}{price}\PY{l+s+s1}{\PYZsq{}}\PY{p}{]}
         
         \PY{n}{param\PYZus{}grid} \PY{o}{=} \PY{p}{\PYZob{}}
             \PY{l+s+s1}{\PYZsq{}}\PY{l+s+s1}{learning\PYZus{}rate}\PY{l+s+s1}{\PYZsq{}}\PY{p}{:} \PY{p}{[}\PY{l+m+mf}{0.02}\PY{p}{,} \PY{l+m+mf}{0.04}\PY{p}{,} \PY{l+m+mf}{0.05}\PY{p}{]}\PY{p}{,}
             \PY{l+s+s1}{\PYZsq{}}\PY{l+s+s1}{max\PYZus{}depth}\PY{l+s+s1}{\PYZsq{}}\PY{p}{:} \PY{p}{[}\PY{l+m+mi}{10}\PY{p}{]}\PY{p}{,}
             \PY{l+s+s1}{\PYZsq{}}\PY{l+s+s1}{max\PYZus{}features}\PY{l+s+s1}{\PYZsq{}}\PY{p}{:} \PY{p}{[}\PY{l+m+mi}{7}\PY{p}{,} \PY{l+m+mi}{8}\PY{p}{,} \PY{l+m+mi}{9}\PY{p}{]}\PY{p}{,}
             \PY{l+s+s1}{\PYZsq{}}\PY{l+s+s1}{min\PYZus{}samples\PYZus{}leaf}\PY{l+s+s1}{\PYZsq{}}\PY{p}{:} \PY{p}{[}\PY{l+m+mi}{2}\PY{p}{]}\PY{p}{,}
             \PY{l+s+s1}{\PYZsq{}}\PY{l+s+s1}{min\PYZus{}samples\PYZus{}split}\PY{l+s+s1}{\PYZsq{}}\PY{p}{:} \PY{p}{[}\PY{l+m+mi}{5}\PY{p}{,} \PY{l+m+mi}{6}\PY{p}{]}\PY{p}{,}
             \PY{l+s+s1}{\PYZsq{}}\PY{l+s+s1}{n\PYZus{}estimators}\PY{l+s+s1}{\PYZsq{}}\PY{p}{:} \PY{p}{[}\PY{l+m+mi}{300}\PY{p}{,} \PY{l+m+mi}{500}\PY{p}{,} \PY{l+m+mi}{700}\PY{p}{,} \PY{l+m+mi}{1000}\PY{p}{]}\PY{p}{,}
             \PY{l+s+s1}{\PYZsq{}}\PY{l+s+s1}{random\PYZus{}state}\PY{l+s+s1}{\PYZsq{}}\PY{p}{:} \PY{p}{[}\PY{l+m+mi}{2}\PY{p}{]}
         \PY{p}{\PYZcb{}}
         
         \PY{n}{gbr} \PY{o}{=} \PY{n}{GradientBoostingRegressor}\PY{p}{(}\PY{p}{)}
         \PY{n}{gbr\PYZus{}grid} \PY{o}{=} \PY{n}{GridSearchCV}\PY{p}{(}\PY{n}{estimator} \PY{o}{=} \PY{n}{gbr}\PY{p}{,} 
                                 \PY{n}{param\PYZus{}grid} \PY{o}{=} \PY{n}{param\PYZus{}grid}\PY{p}{,} 
                                 \PY{n}{cv} \PY{o}{=} \PY{l+m+mi}{3}\PY{p}{,}
                                 \PY{n}{verbose} \PY{o}{=} \PY{l+m+mi}{2}\PY{p}{,}
                                 \PY{n}{n\PYZus{}jobs} \PY{o}{=} \PY{o}{\PYZhy{}}\PY{l+m+mi}{1}\PY{p}{,}
                                 \PY{n}{scoring} \PY{o}{=} \PY{n}{make\PYZus{}scorer}\PY{p}{(}\PY{n}{rmspe\PYZus{}score}\PY{p}{,} \PY{n}{greater\PYZus{}is\PYZus{}better} \PY{o}{=} \PY{k+kc}{False}\PY{p}{)}
         \PY{p}{)}
         
         \PY{c+c1}{\PYZsh{} Treina o modelo com todas as combinacoes}
         \PY{c+c1}{\PYZsh{} do conjunto de param\PYZus{}grid}
         \PY{n}{gbr\PYZus{}grid}\PY{o}{.}\PY{n}{fit}\PY{p}{(}\PY{n}{x}\PY{p}{,} \PY{n}{y}\PY{p}{)}
         
         \PY{c+c1}{\PYZsh{} Define os melhores parametros pro Gradient}
         \PY{c+c1}{\PYZsh{} Boosting Regressor dentro deste conjunto}
         \PY{n}{gbr\PYZus{}grid}\PY{o}{.}\PY{n}{best\PYZus{}params\PYZus{}}
\end{Verbatim}

    \begin{Verbatim}[commandchars=\\\{\}]
Fitting 3 folds for each of 72 candidates, totalling 216 fits

    \end{Verbatim}

    \begin{Verbatim}[commandchars=\\\{\}]
[Parallel(n\_jobs=-1)]: Using backend LokyBackend with 8 concurrent workers.
[Parallel(n\_jobs=-1)]: Done  25 tasks      | elapsed: 10.4min
[Parallel(n\_jobs=-1)]: Done 146 tasks      | elapsed: 94.3min
[Parallel(n\_jobs=-1)]: Done 216 out of 216 | elapsed: 141.9min finished

    \end{Verbatim}

\begin{Verbatim}[commandchars=\\\{\}]
{\color{outcolor}Out[{\color{outcolor}56}]:} \{'learning\_rate': 0.02,
          'max\_depth': 10,
          'max\_features': 7,
          'min\_samples\_leaf': 2,
          'min\_samples\_split': 5,
          'n\_estimators': 1000,
          'random\_state': 2\}
\end{Verbatim}
            
    \hypertarget{verificauxe7uxe3o-dos-paruxe2metros}{%
\subsubsection{Verificação dos
Parâmetros}\label{verificauxe7uxe3o-dos-paruxe2metros}}

    Após a definição dos parâmetros otimizados para o modelo escolhido, é
necessário aplicá-los aos conjuntos de treino e pseudo-teste para
verificação e, posteriormente, gerar o preditor para o conjunto de teste
real.

    Este passo sofreu diversas alterações durante o desenvolvimento do
modelo e está de acordo com a última submissão da autora, em que são
utilizados os parâmetros resultantes do \emph{Grid Search}, sem
tratamento de outliers e com atributos categóricos substituídos por
números inteiros variando de 1 ao número de categorias de cada um.

    \hypertarget{avaliauxe7uxe3o-do-modelo}{%
\paragraph{Avaliação do Modelo}\label{avaliauxe7uxe3o-do-modelo}}

    \begin{Verbatim}[commandchars=\\\{\}]
{\color{incolor}In [{\color{incolor}111}]:} \PY{o}{\PYZpc{}\PYZpc{}time}
          
          \PY{n}{gbr} \PY{o}{=} \PY{n}{GradientBoostingRegressor}\PY{p}{(}\PY{n}{learning\PYZus{}rate} \PY{o}{=} \PY{l+m+mf}{0.02}\PY{p}{,} \PY{n}{max\PYZus{}depth} \PY{o}{=} \PY{l+m+mi}{10}\PY{p}{,} \PY{n}{max\PYZus{}features} \PY{o}{=} \PY{l+m+mi}{7}\PY{p}{,} \PYZbs{}
                                          \PY{n}{min\PYZus{}samples\PYZus{}leaf} \PY{o}{=} \PY{l+m+mi}{2}\PY{p}{,} \PY{n}{min\PYZus{}samples\PYZus{}split} \PY{o}{=} \PY{l+m+mi}{5}\PY{p}{,} \PY{n}{n\PYZus{}estimators} \PY{o}{=} \PY{l+m+mi}{1000}\PY{p}{,} \PYZbs{}
                                          \PY{n}{random\PYZus{}state} \PY{o}{=} \PY{l+m+mi}{2}\PY{p}{)}
          
          \PY{n}{model\PYZus{}analysis}\PY{p}{(}\PY{n}{X\PYZus{}train}\PY{p}{,} \PY{n}{X\PYZus{}test}\PY{p}{,} \PY{n}{y\PYZus{}train}\PY{p}{,} \PY{n}{y\PYZus{}test}\PY{p}{,} \PY{n}{gbr}\PY{p}{,} \PY{l+s+s1}{\PYZsq{}}\PY{l+s+s1}{Gradient Boosting Regression}\PY{l+s+s1}{\PYZsq{}}\PY{p}{)}
\end{Verbatim}

    \begin{Verbatim}[commandchars=\\\{\}]

\#\#\#\#\#\# Gradient Boosting Regression \#\#\#\#\#\#
Score : 0.980995

MSE   : 299954.264518 
MAE   : 270.458491 
RMSE  : 547.680805 
R2    : 0.980995 
RMSPE : 0.089367 
Wall time: 2min 11s

    \end{Verbatim}

    \hypertarget{arquivo-submetido}{%
\paragraph{Arquivo Submetido}\label{arquivo-submetido}}

    \begin{Verbatim}[commandchars=\\\{\}]
{\color{incolor}In [{\color{incolor}112}]:} \PY{o}{\PYZpc{}\PYZpc{}time}
          \PY{n}{x} \PY{o}{=} \PY{n}{train}\PY{o}{.}\PY{n}{drop}\PY{p}{(}\PY{p}{[}\PY{l+s+s1}{\PYZsq{}}\PY{l+s+s1}{price}\PY{l+s+s1}{\PYZsq{}}\PY{p}{]}\PY{p}{,} \PY{n}{axis} \PY{o}{=} \PY{l+m+mi}{1}\PY{p}{)}
          \PY{n}{y} \PY{o}{=} \PY{n}{train}\PY{p}{[}\PY{l+s+s1}{\PYZsq{}}\PY{l+s+s1}{price}\PY{l+s+s1}{\PYZsq{}}\PY{p}{]}
          
          \PY{n}{gbr} \PY{o}{=} \PY{n}{GradientBoostingRegressor}\PY{p}{(}\PY{n}{learning\PYZus{}rate} \PY{o}{=} \PY{l+m+mf}{0.02}\PY{p}{,} \PY{n}{max\PYZus{}depth} \PY{o}{=} \PY{l+m+mi}{10}\PY{p}{,} \PY{n}{max\PYZus{}features} \PY{o}{=} \PY{l+m+mi}{7}\PY{p}{,} \PYZbs{}
                                          \PY{n}{min\PYZus{}samples\PYZus{}leaf} \PY{o}{=} \PY{l+m+mi}{2}\PY{p}{,} \PY{n}{min\PYZus{}samples\PYZus{}split} \PY{o}{=} \PY{l+m+mi}{5}\PY{p}{,} \PY{n}{n\PYZus{}estimators} \PY{o}{=} \PY{l+m+mi}{1000}\PY{p}{,} \PYZbs{}
                                          \PY{n}{random\PYZus{}state} \PY{o}{=} \PY{l+m+mi}{2}\PY{p}{)}
          
          \PY{n}{gbr}\PY{o}{.}\PY{n}{fit}\PY{p}{(}\PY{n}{x}\PY{p}{,} \PY{n}{y}\PY{p}{)}
          \PY{n}{y\PYZus{}pred} \PY{o}{=} \PY{n}{gbr}\PY{o}{.}\PY{n}{predict}\PY{p}{(}\PY{n}{test}\PY{p}{)}
          
          \PY{n}{submission} \PY{o}{=} \PY{n}{pd}\PY{o}{.}\PY{n}{DataFrame}\PY{p}{(}\PY{p}{\PYZob{}}\PY{l+s+s1}{\PYZsq{}}\PY{l+s+s1}{id}\PY{l+s+s1}{\PYZsq{}}\PY{p}{:}\PY{n}{test}\PY{o}{.}\PY{n}{index}\PY{p}{,} \PY{l+s+s1}{\PYZsq{}}\PY{l+s+s1}{price}\PY{l+s+s1}{\PYZsq{}}\PY{p}{:}\PY{n}{y\PYZus{}pred}\PY{p}{\PYZcb{}}\PY{p}{)}
          
          \PY{n}{submission}\PY{o}{.}\PY{n}{to\PYZus{}csv}\PY{p}{(}\PY{l+s+s1}{\PYZsq{}}\PY{l+s+s1}{data/submission.csv}\PY{l+s+s1}{\PYZsq{}}\PY{p}{,} \PY{n}{index} \PY{o}{=} \PY{k+kc}{False}\PY{p}{)}
\end{Verbatim}

    \begin{Verbatim}[commandchars=\\\{\}]
Wall time: 3min 36s

    \end{Verbatim}

    \hypertarget{conclusuxe3o}{%
\section{Conclusão}\label{conclusuxe3o}}

    O trabalho foi de extrema importância para o desenvolvimento do
conhecimento da autora sobre o assunto e gerou motivação para a
realização de um bom modelo. Com o aprendizado obtido a partir dele, foi
possível conceber novas ideias de Aprendizado de Máquina e definir
métricas de melhoria para trabalhos futuros, onde se verificariam a
robustez e capacidade de generalização do modelo de forma ainda mais
específica, utilizando métricas de verificação da variação durante a
validação cruzada, etc.

O resultado obtido pelo código da maneira como ele se encontra, porém,
não foi o melhor de todos. Durante alguns dos testes, a autora trocou os
parâmetros de \emph{min\_samples\_leaf} e \emph{min\_samples\_split}
erroneamente e, nestas condições, o modelo apresentou o melhor resultado
possível e conseguiu atingir o primeiro lugar, público e privado, na
competição (na primeira data de término, às 0h de 13/07). Porém, como
ele foi gerado por uma confusão e não estava de acordo com a lógica
desenvolvida, preferiu-se removê-lo deste relatório e das submissões
consideradas pela avaliação privada.

Uma solução possível para incluí-lo na lógica deste documento seria
colocar a combinação de parâmetros dentro do conjunto considerado pelo
\emph{Grid Search} e rodá-lo novamente para verificar se ele se
encaixaria na melhor escolha de parâmetros possível. Entretanto, como
não houve tempo hábil para realizar este ajuste antes que a competição
acabasse, esta solução será aplicada posteriormente, para melhoria do
modelo.


    % Add a bibliography block to the postdoc
    
    
    
    \end{document}
